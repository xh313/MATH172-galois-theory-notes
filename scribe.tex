\documentclass[12pt]{article}
% Some custom commands to make typing easier. Feel free to add to it


% quick homomorphism
\newcommand{\homo}[1]{\varphi (#1)}
% quick inverse
\newcommand{\inv}[1]{#1^{-1}}
% quick generating set
\newcommand{\genset}[1]{\langle #1 \rangle}
% split problem into multiple parts
\newcommand{\spl}{\rule{\textwidth}{0.4pt}}
% quick normal subgroup
\newcommand{\nsub}{\trianglelefteq}
% quick Z/nZ
\newcommand{\znz}[1]{\mathbb{Z\text{/#1}Z}}

% quick projective surface
\newcommand{\proj}[1]{\mathbb P^{#1}(\mathbb C)}
% quick elliptic curve
\newcommand{\ellip}{E(\mathbb{C})}
% quick ramification
\newcommand{\ram}[1]{e_{#1}}

% todo list
\usepackage[color=yellow]{todonotes}
\newcommand{\tbd}[1]{\todo[inline]{#1}
\addcontentsline{toc}{subsubsection}{TODO}}

% scaling tikz
\usepackage{adjustbox}

% format bullets
% \usepackage[shortlabels]{enumitem}

% Document inherent packages to be loaded
\usepackage{amsmath, amsfonts, amssymb, amsthm, graphicx, url, xcolor, enumerate, tcolorbox}

%geometry helps manage margins, among other things.
\usepackage[rmargin=2in,lmargin=1in,tmargin=1in,bmargin=1in]{geometry}
\newcommand{\sidenote}[1]{\marginpar{\footnotesize \begin{itemize}
    \item[$\leftarrow$]\raggedright #1
\end{itemize}}}

\setlength{\headheight}{16pt}
\setlength{\marginparwidth}{1.5in}
\makeatletter      % make title smaller   
\def\@maketitle{   % custom maketitle 
\begin{center}
    {\Large \@title} \\[0.1in] \@author \\ {\small \@date}
\end{center}  
}
\setcounter{secnumdepth}{0}

\usepackage[colorlinks = true,
            linkcolor = teal,
            urlcolor  = teal,
            citecolor = teal,
            anchorcolor = teal]{hyperref}

\usepackage[all,2cell,ps]{xy}
\usepackage{tikz, tikz-cd}
\usepackage{faktor}
\allowdisplaybreaks
\graphicspath{{Images/}}

\theoremstyle{definition}  % to get rid of the italics
\newtheorem{theorem}{\bt{Theorem}}%[section]
\newtheorem{proposition}[theorem]{\bt{Proposition}}
\newtheorem{corollary}[theorem]{\gt{Corollary}}
\newtheorem{lemma}[theorem]{\bt{Lemma}}
\newtheorem{conjecture}[theorem]{Conjecture}

\newtheorem{defn}{\rt{Definition}}

\newtheorem{eg}{\textcolor{violet}{Example}}
\newtheorem{noneg}[eg]{\textcolor{violet}{Non-example}}

\newtheorem*{rmk}{Remark}


%Import the natbib package and sets a bibliography  and citation styles
\usepackage{natbib}

% Create headers
\usepackage{fancyhdr}
\usepackage{xcolor}
\renewcommand{\headrulewidth}{0pt}
\fancypagestyle{updated}{%
    \fancyhead{MATH 172 Notes \hfill\hfill Xuehuai He}
    \fancyfoot{\hyperlink{toc}{Back to TOC}\hfill\thepage \hfill \textcolor{lightgray}{\today}}
}

% \newcommand{\Belyi}{Bely\u{\i}~}
% \newcommand{\thm}[3]{\vskip 0.2in \begin{center} \fbox{\begin{minipage}{0.8\linewidth} \begin{theorem}[#1] \label{#2} #3 \end{theorem} \end{minipage}} \end{center} \vskip 0.2in}
% \newcommand{\prop}[3]{\vskip 0.2in \begin{center} \fbox{\begin{minipage}{0.8\linewidth} \begin{proposition}[#1] \label{#2} #3 \end{proposition} \end{minipage}} \end{center} \vskip 0.2in}
% \newcommand{\cor}[3]{\vskip 0.2in \begin{center} \fbox{\begin{minipage}{0.8\linewidth} \begin{corollary}[#1] \label{#2} #3 \end{corollary} \end{minipage}} \end{center} \vskip 0.2in}
% \newcommand{\conj}[3]{\vskip 0.2in \begin{center} \fbox{\begin{minipage}{0.8\linewidth} \begin{conjecture}[#1] \label{#2} #3 \end{conjecture} \end{minipage}} \end{center} \vskip 0.2in}

\DeclareMathOperator{\Aut}{Aut}
\DeclareMathOperator{\Gal}{Gal}
\DeclareMathOperator{\Mon}{Mon}
\DeclareMathOperator{\lcm}{lcm}


\newcommand{\rt}[1]{\textcolor{magenta}{#1}}
\newcommand{\gt}[1]{\textcolor{brown}{#1}}
\newcommand{\bt}[1]{\textcolor{cyan}{#1}}

\newcommand{\N}{\mathbb{N}}
\newcommand{\Z}{\mathbb{Z}}
\newcommand{\C}{\mathbb{C}}
\newcommand{\R}{\mathbb{R}}
\newcommand{\Q}{\mathbb{Q}}

\newcommand{\ifnif}{\textit{if and only if }}

\newcommand{\addlink}[1]{\addcontentsline{toc}{subsubsection}{#1}}

\newcommand{\circled}[1]{\begin{tikzpicture}[baseline=(word.base)]
\node[draw, rounded corners, text height=8pt, text depth=2pt, inner sep=2pt, outer sep=0pt, use as bounding box] (word) {#1};
\end{tikzpicture}
}

\newcommand{\ontangent}[1]{
    \spl

\textcolor{darkgray}{\textit{(The following is kind of on a tangent)}}
{#1}
\textcolor{darkgray}{\textit{(Tangent ends here)}}

\spl
}

\usepackage{ulem}
\usepackage{float}
\usepackage{libertinus}

% Make enumeration better
\usepackage{enumitem}
\setlist[itemize]{parsep=0pt,topsep=0pt}
\setlist[enumerate]{parsep=0pt,topsep=0pt}

% \definecolor{pomona}{HTML}{#0057b8}

\parskip = .2in
\parindent = 0in

% clever ref (load last)
\usepackage[capitalise]{cleveref}
\newcommand{\fref}{\cref}
\newcommand{\Fref}{\Cref}
\newcommand{\prettyref}{\cref}
\newcommand{\newrefformat}[2]{}

 %Cleveref definitions
\crefname{lemma}{Lemma}{Lemmas}
\crefname{thm}{Theorem}{Theorems}
\crefname{defn}{Definition}{Definitions}
\crefname{notn}{Notation}{Notations}
\crefname{const}{Construction}{Constructions}
\crefname{prop}{Proposition}{Propositions}
\crefname{rem}{Remark}{Remarks}
\crefname{cor}{Corollary}{Corollaries}
\crefname{equation}{Equation}{Equations}
\crefname{ex}{Example}{Examples}

%=============

\begin{document}
\title{MATH172 Galois Theory Notes}
\author{Xuehuai He}
\maketitle

\hypertarget{toc}{}
{\parskip=0.05in
\tableofcontents}

\spl

\newpage
\pagestyle{updated}

\section{Rings! Or why $x^2-2$ has roots.}

\defn{A \textbf{ring} is a set $R$ together with associative binary \textit{operations}\sidenote{map from $R\times R\mapsto R$} $+$ and $\times$ s.t.:\begin{itemize}
    \item $(R,+)$ is an \textbf{abelian} group with identity 0
    \item There exists $1\in R$ s.t. $r\times1=1\times r=r$\sidenote{this is optional}
    \item $r(s+t)=rs+rt$ and $(s+t)r=sr+tr \qquad \forall s,r,t\in R$
\end{itemize}}
\addcontentsline{toc}{subsubsection}{Ring}

\proposition{$0\times 1=0$ (in fact, $0\times r = 0\; \forall \;r\in R$)

\proof{Try it! \qed}
}

\defn{If $\times$ is commutative, then $R$ is a commutative ring.}
\noneg $\N$ is not a ring.
\eg{$\Z \subseteq \mathbb{Q}\subseteq\R\subseteq\C$ are all rings;
\begin{itemize}
    \item $\znz{\textit{n}}$ is a finite ring
    \item $M_n(\R)$, the set of $n\times n$ real matrices, is a \textbf{noncommutative} ring
    \item Polynomial ring: $\mathbb{Q}[x]=\left\{a_0+a_1x+\dots+a_nx^n \mid a_i\in \mathbb{Q}\right\}$\sidenote{square brackets just mean "polynomials in..."} is a commutative ring
    \item $\mathbb{Z}[\sqrt{-5}]=\left\{a+b\sqrt{-5} \mid a,b\in \mathbb{Z}\right\}$ is a commutative ring
\end{itemize}
}

\spl

\begin{tcolorbox}
    Phase I plan:
    $$ID\supsetneq UFD\supsetneq PID\supsetneq ED \supsetneq Fields$$
\end{tcolorbox}
\addcontentsline{toc}{subsection}{Phase I: ID, UFD, PID, ED, Fields}

\spl

\defn{Suppose R is a ring and $a,b\in R$ with $ab=0$ but $a,b\neq 0$; then $a,b$ are called \textbf{zero divisors}.}
\addcontentsline{toc}{subsubsection}{Zero divisors}

\eg{\hfill
    \begin{itemize}
    \item In $\znz{6}$, $\bar4\times\bar3=\bar0$
    \item In $M_2(\mathbb{R})$, $\begin{pmatrix}
        1&0\\0&0
    \end{pmatrix}\begin{pmatrix}
        0&0\\0&1
    \end{pmatrix}=\begin{pmatrix}
        0&0\\0&0
    \end{pmatrix}$
\end{itemize}}

\defn{A commutative ring without zero divisors is called an \underline{integral domain} (ID)}
\addcontentsline{toc}{subsubsection}{Integral domain}

Why do we want ID? \textbf{Cancellation properties}.
\begin{itemize}
    \item If $R$ is an ID, $a,b,c\in R$, $a\neq 0$ and $ab=ac$, then \begin{equation*}
        ab-ac=0\implies a(b-c)=0 \implies b-c=0 \implies b=c
    \end{equation*}
\end{itemize}

\defn{Suppose $R$ is an ID. An element $a\in R$ is called a \textbf{unit} if $a\neq 0$ and there exists $b\in R$ s.t. $ab=1$ \sidenote{notation: $b=\inv{a}$}. 

An element $r\in R$ is called \textbf{irreducible} if $r\neq 0$, $r$ is NOT a unit, and whenever $r=ab$ for some $a,b\in R$ then $a$ or $b$ must be a unit.}\label[definition]{unit-irreducible}
\addcontentsline{toc}{subsubsection}{Unit, irreducibles}

\begin{itemize}
    \item If $r$ and $s$ are irreducibles with $r=us$, then $r$ and $s$ are called \textbf{associates}.
\end{itemize}

\eg \hfill
\begin{itemize}
    \item All ``prime integers'' are irreducibles in $\mathbb{Z}$;
    \item 2,3, $1+\sqrt{-5}, 1-\sqrt{-5}$ are irreducibles in $\mathbb{Z}[\sqrt{-5}]$.\begin{itemize}
        \item Note: $2\times 3= (1+\sqrt{-5})(1-\sqrt{-5})=6$ says that 6 can be factored in more than one way. This means that $\mathbb{Z}[\sqrt{-5}]$ is NOT an UFD. 
    \end{itemize}
\end{itemize}

\defn{An integral domain $R$ is called a \underline{unique factorization domain} (UFD) if each nonzero, nonunit $a\in R$ can be written as a product of irreducibles \textbf{in a unique way} up to associates.}
\addcontentsline{toc}{subsubsection}{Unique factorization domain}

If $a$ is a nonzero, nonunit element of UFD $R$ and $a=r_1r_2\dots r_m=s_1\dots s_n$ where $r_i,s_j$ are irreducible, then after reordering $r_i=u_is_i$ for any $i$ and units $u_i$, and $m=n$.\sidenote{After reordering, there are the same amounts of factors and all factors are the same up to units.}

\defn{Suppose $R$ is a comm ring. A subset $I\subseteq R$ is called an \textbf{ideal} if $(I,+)\leq (R,+)$ and $ir, ri\in I$ for all $i\in I$ and for all \circled{$r\in R$}.}
\addcontentsline{toc}{subsubsection}{Ideals}

Why do we want ideals? Such that $R/I$ is a well-defined ring.
\eg \{0\} and $R$ are ideals of $R$.
\eg If $R$ is commutative and $a\in R$, then $(a)=\{ar\mid r\in R\}$ is called the \textbf{principal ideal} generated by $a$.\sidenote{Prove this (be convinced)!\\ Also known as $aR$.}

\defn{A \textbf{principal ideal domain} is an integral domain where all ideals are principal ideals.}
\addcontentsline{toc}{subsubsection}{Principal Ideal Domain}

\eg The only ideals of $(\Z,+)$ are of the form $n\Z=(n)$.\sidenote{Ideals generated by $n$}
\noneg $\Z[x]$ is a UFD but NOT a PID because the ideal $(2,x)=\{2r+xs\mid r,s\in \Z[x]\}$ is not principal.\sidenote{Observe that $(2,x)$ is an ideal made of polynomials with even constant terms. This cannot be principal, since if we only have $2$ and not $x$, we do not have nonzero polynomials with zero const terms.}

\lemma If $I\subseteq R$ is an ideal and $1\in I$, then $I=R$.\label[lemma]{lemma:1-in-I-then-I-is-R}
\proof Try it!

\proposition If $I\subseteq R$ is an ideal containing a unit of $R$ then $I=R$.
\proof If $u\in I$ is a unit then $\inv{u}\in R$, so $u\inv u=1\in I$. Then the result follows from \cref{lemma:1-in-I-then-I-is-R}. \qed

\defn A \textbf{field} is a commutative ring whose \uline{each nonzero element is a \textit{unit}}.

\begin{corollary}
    If $R$ is an ID whose ideals are $(0)$ and $R$, then $R$ is a \textbf{field}.
\end{corollary}\sidenote{The converse is also true. \textbf{The only ideals in a field are 0 and the field.}}
\proof Suppose $a\in R\backslash \{0\}$ and consider $(a)$. Since $a\in (a),\, (a)=R$. Hence, we must have that $1\in(a)$, which means $1=ar$ for some $r\in R$.\qed

\begin{defn}
    Suppose $R$ is an integral domain. A \textit{proper} ideal $P\subsetneq R$ is called \textbf{prime} of whenever $ab\in P$ for some $a,b\in R$, then $a$ or $b\in P$. 
\end{defn}
\addcontentsline{toc}{subsubsection}{Prime Ideal}

\noneg $(6)$ is not a prime ideal of $\Z$ since $2\times 3\in (6)$ but neither $2,3\notin (6)$.
\noneg $(2)$ is not a prime ideal of $\Z[\sqrt{-5}]$ since $6\in (2)$\sidenote{Observe that in $\Z[\sqrt{-5}]$, we have $6=(1+\sqrt{-5})(1-\sqrt{-5})=2\times 3$, so it is not a UFD!}, but we observe that $6=(1+\sqrt{-5})(1-\sqrt{-5})$ while $1\pm \sqrt{-5}\notin (2)$.
\eg (2) is a prime ideal of $\Z$.

\begin{defn}
    A proper ideal $M\subsetneq R$ is called \textbf{maximal} if whenever $I\subseteq R$ such that $M\subseteq I\subseteq R$ is an ideal containing $M$, then either $I=M$ or $I=R$.
\end{defn}
\addcontentsline{toc}{subsubsection}{Maximal Ideal}
\begin{proposition}
    Every proper ideal is contained in \textbf{a} maximal ideal.\sidenote{This might not be unique in non-local rings.}
\end{proposition}
\begin{proof}
    Axiom of choice.
\end{proof}

\begin{proposition}\label[prop]{mod-zero-ideal}
    Suppose $R$ is a commutative ring.\begin{itemize}
        \item (0) is prime \textit{if and only if} $R$ is an integral domain.\sidenote{By def of prime, if $ab=0$, then either $a=0$ or $b=0$, which means there are NO zero divisors.}
        \item (0) is maximal \textit{if and only if} $R$ is a field.
    \end{itemize}
\end{proposition}

\ontangent{
    \defn A commutative ring $R$ with unity is called \textbf{Noetherian} if, whenever $I_1\subseteq I_2\subseteq \dots$ is an ascending sequence of (proper) ideals of $R$, there exists an $n>0$ such that $I_n=I_{n+1}=\dots$ are the same ideals thereafter.\sidenote{The chain stops ascending!}
\addcontentsline{toc}{subsubsection}{Noetherian Rings}

\begin{theorem}
    $R$ is Noetherian \textit{if and only if} all ideals of $R$ are finitely generated.
\end{theorem}

\begin{corollary}
    All Principal Ideal Domains are Noetherian.\sidenote{Since all ideals are generated by 1 elt.}
\end{corollary}
}

\defn Suppose $R$ is a commutative ring with $1\neq 0$ and $I\subseteq R$ is an ideal. Then the \underline{\textbf{quotient ring}} of $R$ by $I$ is the set $$R/I=\{r+I\mid r\in R\}$$ with addition and multiplication defined representative-wise.
\addcontentsline{toc}{subsubsection}{Quotient Rings}

\rmk The \textbf{coset criterion} of ideals: let $I$ be an ideal; the cosets $r+I$, $s+I$ are the same \textit{if and only if} $r-s\in I$.

\eg \hfill
\begin{itemize}
    \item In $\Z/(6)$ aka. $\znz{6}$, we have $2+(6)=\{\dots,-10,-4,2,8,14,\dots\} = 26+(6)$ due to $2-26\in (6)$;
    \item In $\mathbb{Q}[x]/(x^2-2)$, we have $$\{3x^2-47x+1+q(x)(x^2-2)\mid q(x)\in \mathbb{Q}[x]\} = \{-47x+7+q(x)(x^2-2)\mid q(x)\in \mathbb{Q}[x]\}$$ due to $3x^2-47x+1-(-47x+7)\in (x^2-2)$.
\end{itemize}

\rmk Let $I$ be an ideal of $R$. Then \underline{$(I,+)\nsub (R,+)$}. 

\defn $R/I$ is a group under $(r+I)+(s+I)=(r+s)+I$ and the operation + is well-defined. We also define that $(r+I)(s+I)=(rs)+I$. We claim that multiplication in $R/I$ is also well-defined.
\begin{proof}
    Let $r_1+I=r_2+I$ and $s_1+I=s_2+I$. By coset criterion, $r_1-r_2=i$, $s_1-s_2=j$ for some $i,j\in I$. Hence $r_1s_1=(r_2+i)(s_2+j)=r_2s_2+is_2+jr_2+ij$ where the latter three terms are all in the ideal $I$. Thus, $(r_1s_1)+I=(r_2s_2)+I$.
\end{proof}

From $R$, $R/I$ inherits nice properties:
\begin{itemize}
    \item $0+I=0_{R/I}$
    \item $1+I=1_{R/I}$
    \item Multiplication is commutative and distributive over addition in $R/I$, so it is also a comm. ring with identity.
\end{itemize}

\defn A function $\varphi: R\rightarrow S$ between rings is called a \textbf{ring homomorphism} if the following are satisfied:
\begin{itemize}
    \item $\homo{r_1+r_2}=\homo{r_1}+\homo{r_2}$
    \item $\homo{r_1r_2}=\homo{r_1}\homo{r_2}$
\end{itemize}
\addcontentsline{toc}{subsubsection}{Ring homomorphism}

\begin{theorem}\label{FIT}
    \underline{First ring isomorphism theorem}\sidenote{Observe that kernels are ideals! And ideals are kernels of some homomorphism too.}

    If  $\varphi: R\mapsto S$ is a ring homomorphism, then $R/\ker(\varphi)\cong \homo{R}$.
\end{theorem}
\addcontentsline{toc}{subsubsection}{First ring isomorphism theorem}

\eg If $R$ is a ring and $I$ is an ideal, then $\pi : R\rightarrow R/I$ where $r\mapsto r+I$ is a surjective homomorphism where $\ker(\pi)=I$. This is the \textit{canonical projection} onto $R/I$.

\begin{corollary}
    If $I$ is a maximal ideal, then $R/I$ is a field.
\end{corollary}\sidenote{The \ifnif version comes in \cref{quotient-max-field}.}

Recall \cref{mod-zero-ideal}. We now have a stronger statement:

\begin{proposition}
    Suppose $R$ is a commutative ring \& $P\subseteq R$ is an ideal. Then $R/P$ is an integral domain \textit{if and only if }$P$ is prime.
\end{proposition}
\addcontentsline{toc}{subsubsection}{Quotient by prime ideal is ID}
\begin{proof}
    $R/P$ is an integral domain \ifnif whenever $(a+P)(b+P)=0_{R/P}$ then one of $a+P$ or $b+P$ must already be $0_{R/P}$. This happens \ifnif whenever $ab+P=P$ then $a+P$ or $b+P$ in $P$, which happens \ifnif whenever $ab\in P$ then one of $a,b\in P$, which is the definition of a prime ideal.
\end{proof}

\eg The map $\varphi: \Z[x]\to \Z$ where $p(x)\mapsto p(0)$ is a surjective ring homomorphism with $\ker(\varphi)=(x)$. By the the First Isomorphism \cref{FIT}, $\Z[x]/(x)\cong \Z$. \sidenote{btw, $(x)\subsetneq (x,2)$. the latter is the set of polynomials whose \uline{constant term is even}, so it is also a proper ideal of $\Z[x]$. This is an excellent example where Prime $\nRightarrow$ Maximal.} As such, we conclude that $(x)$ is a prime ideal since $\Z$ is an integral domain.

\begin{lemma}
    Suppose $R$ is a comm. ring with $M\subseteq R$ being an ideal. \\
    There is a bijective correspondence between the ideals of $R/M$ and the ideals of $R$ containing $M$.
\end{lemma}\label[lemma]{4th-isomorphism-thrm}
\begin{proof}
    Consider the projection $\pi:R\to R/M$ where $r\mapsto r+M$. It is enough to show: \sidenote{To see why this is okay, see Homework 2 Sec. 7.3 P. 24}\begin{align*}
        \pi(\inv{\pi}(J))&=J && \text{for all ideals } J\subseteq R/M \text{, and}\\
        \inv\pi({\pi}(I))&=I && \text{for all ideals } M\subseteq I\subseteq R
    \end{align*}
To prove the first statement, observe that, if $J$ is an ideal of $R/M$, then $\inv{\pi}(J)=\{r\in R\mid r+M\in J\}$ and so $$\pi(\inv{\pi}(J))=\{\pi(r)\in R\mid r+M\in J\}= \{r+M\mid r+M\in J\}= J$$

Next, to prove the second statement, first suppose $M\subseteq I\subseteq R$ is an ideal. Let $a\in I$. Then $a+M\in \{\alpha+M\mid \alpha\in I\}=\pi(I)$. This implies that $a\in \inv{\pi}(\pi(I))$, and so \dashuline{$I\subseteq \inv{\pi}(\pi(I))$}. 

Conversely, suppose $r\in \inv{\pi}(\pi(I))$. This is the same as saying $\pi(r)=r+M\in \pi(I)=\{\alpha+M\mid \alpha \in I\}$. Hence, for any $r\in \inv{\pi}(\pi(I))$, there exists some $a\in I$ such that $r+M=a+M$. Thus, $r-a\in M\subseteq I$ by coset conditions. Since $a\in I$, we have $a+(r-a)\in I$, meaning that $r\in I$ for any $r\in \inv{\pi}(\pi(I))$. This means that \dashuline{$\inv{\pi}(\pi(I))\subseteq I$}. 

Hence, \uline{$I= \inv{\pi}(\pi(I))$}. 

Consequently, for any ideals $J\subseteq R/M$, we know that $\inv{\pi}(J)\subseteq R$ is an ideal containing $M$\sidenote{Think about why this contains $M$!}. And if $M\subseteq I\subseteq R$ is an ideal, we know $\pi(I)\subseteq R/M$ is an ideal. Since $\pi(\inv{\pi}(J))= J$ and $I= \inv{\pi}(\pi(I))$ for any $I,J$, the correspondence is a bijection.
\end{proof}

\begin{proposition}
    Suppose $R$ is a comm. ring with an identity and $I\subseteq R$ is an ideal. Then $R/I$ is a field \ifnif $I$ is maximal.
\end{proposition}\label[prop]{quotient-max-field}
\addcontentsline{toc}{subsubsection}{Quotient by maximal ideal is field}
\begin{proof}
    If $I$ is maximal, then there are no other proper ideals strictly containing $I$. Hence, by \cref{4th-isomorphism-thrm}, we have that $R/I$ only have ideals $(0)$ and $R/I$ itself. This happens \ifnif $R/I$ is a field.
\end{proof}

\begin{corollary}
    If $R$ is a commutative ring with identity and $M\subseteq R$ is maximal, then $M$ is prime.\sidenote{Hence \uline{maximal implies prime}, but prime does not necessarily implies maximal.}
\end{corollary}
\begin{proof}
    Maximal $\implies$ quotient is a field $\implies$ quotient is an ID $\implies$ prime.
\end{proof}

\defn An integral domain $R$ is an \textbf{Euclidean domain} if there exists a norm $N: R\to \Z_{\geq 0}$ with $N(0)=0$ such that for all $a,b\in R$ with $b\neq 0$, there exists $q,r\in R$ for which $$a=bq+r$$ with $N(r)<N(b)$ or $r=0$.
\addcontentsline{toc}{subsubsection}{Euclidean Domain}

\eg $\Z$ is a ED with $N(a)=|a|$.
\eg $\Q[x]$ is a ED with $N(p(x))=\deg (p(x))$.
\eg Every field $F$ is a ED with $N(a)=0\;\forall\; a\in F$.\sidenote{Because in a field everything divides!}
\noneg $\Z\left[\frac{1+\sqrt{-19}}{2}\right]$ is a PID that is not an ED.\sidenote{This is one of the only good examples!}

\subsubsection{Why do we care about Euclidean domains?}
\rmk \underline{Greatest common divisors} exist and are relatively quick to compute. \sidenote{Using recursive application of Euclidean algorithm.}

\defn If $a,b\in R$, then $\gcd (a,b)=c$ means \begin{enumerate}
    \item $c$ divides $a$ and $b$; that is, $a=cr, b=cs$ for some $r,s\in R$
    \item If $c'\in R$ with $c'|a$ and $c'|b$, then it must be true that $c'|c$.\sidenote{All other common divisors divide the gcd.}
\end{enumerate}
\addcontentsline{toc}{subsubsection}{Greatest common divisors \& Euclidean algorithm}

\eg Say we want to compute the gcd of 47 and 10.\sidenote{This is a much faster algorithm than factoring!}
\begin{align*}
    47 &= 4\times 10 + 7\\
    10 &= 1\times 7 + 3\\
    7 &= 2\times 3 + \circled{1} && \leftarrow \text{circled is gcd}(47,10)\\
    3 &= 3\times 1 && \leftarrow \text{final line with no remainders}
\end{align*}

This also works for finding gcds in $\Q[x]$ with polynomials long division and norm $\deg (p(x))$.

\rmk If $F$ is a field, then $F[x]$ is a Euclidean domain.\sidenote{Just use long division!}

\rmk Euclidean domains are PIDs.
\begin{proof}
    Suppose $R$ is a ED and $I\subseteq R$ is an idea;. Consider $\{N(a)\mid a\in I\backslash \{0\}\}$. This set has a \underline{minimal element} by properties of natural numbers (or is an empty set \ifnif $I=(0)$).

    Let $d\in I$ be an element of \underline{minimum norm} (hence $N(d)\leq N(a)$ for all $a\in I$). We claim that \underline{$(d)=I$}. Proof:

    Since $d\in I$, we have $rd\in I$ for any $r\in R$. This implies that $(d)\subseteq I$.

    Then let $a\in I$. Since $R$ is a ED, we first assumes that there exists $q,r\in R$, $r\neq 0$ such that $a=qd+r$ and $N(r)<N(d)$. But we notice that $r=a-qd$ must be in $I$ as both $a, qd\in I$, contradicting the minimality of $N(d)$. Thus, it must be that $r=0$. This implies $a=qd$ and thus $a\in (d)$ for all $a\in I$. \underline{Consequently, $I\in (d)$}, and therefore $I=(d)$.
\end{proof}

\defn Suppose $R$ is an integral domain and $p\in R\backslash \{0\}$. Then $p$ is a \textbf{prime element} if $(p)$ is a prime ideal.
\addlink{Prime elements}

\begin{proposition}
    An element $p\in R$ is prime \ifnif whenever $p|ab$ then $p|a$ or $p|b$.
\end{proposition}
\begin{proof}
    $p$ is prime means that $(p)$ is a prime ideal. This is true \ifnif whenever $ab\in (p)$ then $a\in (p)$ or $b\in (p)$. This is the same as saying if $ab=kp$ for some $k\in R$ then $a=lp$ or $b=lp$ for some $l\in R$. This is to say that whenever $p|ab$ then $p|a$ or $p|b$.
\end{proof}

\begin{proposition}
    In an integral domain, all prime elenments are irreducibles.
\end{proposition}
\addlink{Prime implies irreducible in ID}
\begin{proof}
    Suppose $R$ is an ID and $p\in R$ is prime. If $p=ab$ for some $a,b$ in $R$, then, WLOG, $p|a$. That is, $a=pk$ for some $k\in R$. Hence, $p=pkb$. Since in an ID cancellation rule holds, $kb=1$, meaning that $b$ is a unit. Thus, $p$ is irreducible by definition \cref{unit-irreducible}.
\end{proof}

\begin{proposition}
    In PIDs, all \textit{nonzero }prime ideals are maximal.
\end{proposition}
\addlink{Prime implies maximal in PID}
\begin{proof}
    Suppose $R$ is a PID and $(p)\subseteq R$ is a prime ideal. If $(p)\subseteq (m)\subseteq R$ is an ideal, then $p\in (p)\subseteq (m)$ hence $p=rm$ for some $r\in R$. Since $p|rm$, we have $p|r$ or $p|m$.

    If $p|r$, this implies that $r=pk$ for some $k\in R$. Substituting into $p=rm$, we get $p=pkm$. By cancellation, we get $km=1$, meaning that $m$ is a unit. Hence, $(m)=R$.

    If $p|m$, we have $m=pl$ for some $l\in R$, meaning that $m\in (p)$. Hence, $(m)\subseteq (p)$, but we also defined that $(p)\subseteq (m)$, so $(m)=(p)$.

    Therefore, $(p)$ has to be the maximal ideal.
\end{proof}

\begin{proposition}
    In an UFD, irreducible implies prime.
\end{proposition}
\begin{proof}
    Let $R$ be a UFD and $p\in R$ be irreducible. Let $a,b\in R$ such that $p|ab$. Hence, $pr=ab$ for some $r\in R$. Since $R$ is a UFD, let $a=q_1\dots q_n, b= s_1\dots s_m$ be the factorization. Since the factorizations are unique and each of the $q_i, s_j$ are irreducible, if $p|ab$, then $p$ must be an associate with one of the $q_i, s_j$. Therefore, either $p|a$ or $p|b$, implying prime.
\end{proof}
\addlink{Irreducible implies prime in UFD}

\eg $\Q$ is a field, so $\Q[x]$ is a ED. Since EDs are UFDs, irreducible $\implies$ prime. We see that $x^2-2\in \Q[x]$ is an irreducible element, which means that $(x^2-2)$ is a prime ideal, meaning that it is a maximum ideal, meaning that $\Q[x]/(x^2-2)$ is a field. We observe that it is a field containing $\Q$ and $(\sqrt{2})$.\sidenote{In fact, this is the smallest field containing $\Q$ and $(\sqrt{2})$.}

\begin{lemma}
    In a PID, irreducible elements are prime.
\end{lemma}
\begin{proof}
    \parskip =0pt
    Suppose $p\in R$ is irreducible in the principal ideal domain $R$. If $p|ab$ for some $a,b\in R$, we want to show that either $p|a$ or $p|b$, hereby showing that $p$ is prime. Hence, we consider the ideal $(a,p)=d$, which is necessarily principal for some $d\in R$. Since $a,p\in (d)$, we have $a=dr$ and $p=ds$ for some $r,s\in R$. As $p$ is irreducible, we get that one of $d$ and $s$ is a unit. 
    
    We first assume that $s$ is a unit, in which case $d=p\inv s$, and so $a=p\inv sr$ implying that $p|a$. 
    
    In another case, $d$ is a unit, in which case $(a,p)=(d)=R$ and so $1=ak+pl$ for some $k,l\in R$. Multiplying by $b$, we get $b=abk+pbl$. Since $p|ab$, we have $b=abk+pbl=pmk+pbl$ for some $m\in R$. Hence, $b=p(mk+bl)$, meaning that $p|b$.

    Therefore, whenever $p|ab$, either $p|a$ or $p|b$. Hence, in a PID, $p$ is prime whenever it is irreducible.
\end{proof}

\begin{proposition}
    PIDs are UFDs.
\end{proposition}\addlink{PIDs are UFDs}
\begin{proof}
    Suppose $R$ is a PID and $a\in R$ is nonzero, nonunit. If $a$ is irreducible, we are done. If not, we write $a=p_1q_1$ for some $p_1,q_1\in R$ nonunit. If $p_1,q_1$ are irreducibles, we are done. If not, then WLOG say $q_1=p_2q_2$ for some nonunits $p_2,q_2$. We would like to show that this splitting process terminates.

    Observe that $(q_1)\subseteq (q_2)$ since $q_2|q_1$. Hence, the chain of splitting results in the chain of ideals $(q_1)\subseteq (q_2)\subseteq (q_3)\subseteq \dots$.

    Now consider the ideal\footnote{The proof that this is an ideal is as follows:
    
    We first prove that $\bigcup_{n=1}^\infty I_n$ is a subgroup of $R$ under addition. Let $r,s\in \bigcup_{n=1}^\infty I_n$, where $r\in I_k$ and $s\in I_{k+i}$ for some $k,i\in \N$. Since $I_1\subseteq I_2\subseteq \dots$ are ideals of $R$, we know that $r\in I_k$ implies that $r\in I_{k+i}$. Thus, $r-s\in I_{k+i}$ due to $I_{k+i}$ being an ideal. As $I_{k+i}\subseteq \bigcup_{n=1}^\infty I_n$, we have $r-s\in \bigcup_{n=1}^\infty I_n$, which means that $\bigcup_{n=1}^\infty I_n$ is closed under additive inverse. Hence, \dashuline{$\bigcup_{n=1}^\infty I_n$ is a subgroup of $R$ under addition.}

Then, we prove that for any $t\in R$, $r\in \bigcup_{n=1}^\infty I_n$, we would have $tr,rt\in \bigcup_{n=1}^\infty I_n$. Since $r\in \bigcup_{n=1}^\infty I_n$, it must be true that $r\in I_k$ for some $k\in \N$. Hence, $tr,rt \in I_k$  due to $I_{k}$ being an ideal. Therefore, \dashuline{$tr,rt \in \bigcup_{n=1}^\infty I_n$ for any $t\in R$, $r\in \bigcup_{n=1}^\infty I_n$.}

In conclusion, since $\bigcup_{n=1}^\infty I_n$ is a subgroup of $R$ under addition with the property that $tr,rt \in \bigcup_{n=1}^\infty I_n$ for any $t\in R$, $r\in \bigcup_{n=1}^\infty I_n$, it is an ideal of $R$.\qed
} $\bigcup_{i=1}^{\infty}(q_i)$. Since this is a PID, we have $\bigcup_{i=1}^{\infty}(q_i)=(q)$ for some $q\in R$. Since $q\in \bigcup_{i=1}^{\infty}(q_i)$, it is contained in some $(q_n)$ for some $n\geq 1$. This implies that $(q)\subseteq (q_n)$, but we also know that $(q_n)\subseteq (q)$, hence $(q)=(q_n)$. Hence, this process terminates, and there exists an $n$ in this chain such that $q_n$ is irreducible. Therefore, $R$ is a \uline{factorization domain}.

Now we want to prove the \uline{uniqueness}. That is, if $p_1\dots p_n = q_1\dots q_m$ for irreducibles $p_i,q_j$ and $n\leq m$ WLOG, then we want to show that $m=n$ and that $p_i=u_iq_i$ with units $u_i$ up to reordering for all $i$. We do so by induction on $n$.
\begin{itemize}[align=left]
    \item[(\textit{Base case})] If $p_1=q_1\dots q_m$ and $p_1$ irreducible, then $q_2\dots q_m$ are all units. Hence, $m=1$ and $p_1=q_1$.
    \item[(\textit{Inductive step})] Say we have already proven the statement for $n=k$. Then consider $p_1p_2\dots p_{k+1}=q_1q_2\dots q_m$. Since $R$ is a PID where irreducible implies prime, $p_1$ is a prime element dividing the product of primes $q_1q_2\dots q_m$, so we say WLOG $p_1|q_1$. This means that $q_1=u_1p_1$ for some $u\in R$, but since $q_1$ is not reducible, it forces $u_1$ to be a unit. Hence, we apply cancellation on both sides and get $p_2\dots p_{k+1}=(u_1q_2)\dots q_m$.
\end{itemize}
By inductive hypothesis, $m-1=k$ and $p_i, q_i$ are associates up to reordering for any $i$. Hence, the factorization must be \uline{unique}.
\end{proof}

\spl
\begin{tcolorbox}
    We have shown:
    \[\begin{tikzcd}[ampersand replacement=\&,cramped,sep=tiny]
        ID \& \supsetneq \& UFD \& \supsetneq \& PID \& \supsetneq \& ED \& \supsetneq \& Fields \\
        {\text{Non-eg: }\mathbb{Z}/6\mathbb{Z}} \&\& {\mathbb{Z}[\sqrt{-5}]} \&\& {\mathbb{Z}[x]} \&\& {\mathbb{Z}\left[\frac{1+\sqrt{-19}}{2}\right]} \&\& {\mathbb{Z}} \\
        {} \&\&\&\&\&\&\&\& {} \\
        \&\& {} \&\&\&\&\&\& {} \\
        \&\&\&\& {} \&\&\&\& {}
        \arrow["{Maximal\implies Prime\implies Irreducible}"{description}, maps to, no head, from=3-1, to=3-9]
        \arrow["{Prime \iff Irreducible}"{description}, maps to, no head, from=4-3, to=4-9]
        \arrow["{Prime\iff Maximal}"{description}, maps to, no head, from=5-5, to=5-9]
    \end{tikzcd}\]
\end{tcolorbox}
\addcontentsline{toc}{subsection}{Phase I summary}
\spl

\section{Field extensions}
We observe that the polynomial $x^2-2\in \Q [x]$ is irreducible. If we have $x^2-2=p(x)q(x)$ where $p,q$ nonunits, then $\deg(p)+\deg(q)=2$ and we cannot have any $0+2$ combinations due to constants being units, we only have $x^2-2=(x-\sqrt{2})(x+\sqrt{2})$, but $x±\sqrt{2}\notin \Q[x]$!

Since $\Q[x]$ is a UFD, the irreducible element $(x^2-2)$ is prime, and since $\Q[x]$ is a PID, $(x^2-2)$ is maximal which means that $\Q[x]/(x^2-2)$ is a field.

\spl
\addcontentsline{toc}{subsection}{Phase II: Field extensions}
\begin{tcolorbox}
    Phase II plan: Field extensions!
\end{tcolorbox}
\spl

Suppose $F$ is a field and $p(x)\in F[x]$ nonzero. Recall that $F[x]$ is a ED with the norm function $\deg (a(x))$ and long division of polynomials. \sidenote{$a(x)$ is a coset rep}Let $a(x)+(p(x))\in F[x]/(p(x))$. By the division algorithm\sidenote{We can do division algorithm since this is an ED}, we have $a(x)=p(x)q(x)+r(x)$ for $q(x),r(x)\in F[x]$ and $\deg(r(x))<\deg(p(x))$ or $r(x)$ is the zero polynomial.

Now we see that since $a(x)-r(x)\in (p(x))$, they are in the same coset! Hence $a(x)+(p(x))=r(x)+(p(x))$. We observe that \uline{every element of $F[x]/(p(x))$ can be represented by a polynomial of a degree less than $\deg(p(x))$}. In other words, if $\deg(p(x))=n$, then $F[x]/(p(x))$ is of the form \sidenote{\scriptsize The expression under the bar functions like $r(x)$! Also note that span is just the set of linear combinations.}
\begin{align*}
    F[x]/(p(x))&=\left \{\,\overline{a_0+a_1x+\dots +a_{n-1}x^{n-1}}\bigm\vert a_0,a_1,\dots,a_{n-1}\in F \right \}\\
    &= \text{Span}_F\{\bar 1, \bar x, \dots, \bar{x^{n-1}}\}
\end{align*}\sidenote{Why is this not the vector space over $F/(p(x))$ but just $F$? See the next paragraph.}
In fact, $F[x]/(p(x))$ is (partly) just a \textbf{vector space} over $F$\dots

We shall observe that it does not matter if we are using $F$ or $\bar F$.\\
Consider $\varphi: F\hookrightarrow F[x]/(p(x))$ where $a\mapsto \bar a$. We observe this is an \textbf{injective} map: whenever $\deg(p(x))=n>0$, we have $\homo{a}=\homo{b}$ \ifnif $\bar a=\bar b$, which happens \ifnif $a-b\in (p(x))$; but the difference of two constants always have deg 0 and cannot be in $(p(x))$ unless it is a straight zero, which tells us that $\bar a = \bar b$ \ifnif $a=b$. \uline{In other words, $F[x]/(p(x))$ contains an isomorphic copy of $F$, its field of scalars!} Namely, \circled{$F\cong \homo{F}=\{\bar a \in F[x]/(p(x))\mid a\in F\}$}.

\dots Hence, $F[x]/(p(x))$ is a vector space \textbf{of dimension $n$} over the scalar field $F$ that also contains an isomorpic copy of $F$.
\addlink{$F[x]/(p(x))$ contains a copy of $F$}

Moreover, \uline{if $p(x)$ is irreducible}, then $(p(x))$ is prime since this is an ED, and hence, it is also a maximal\sidenote{all thanks to Euclidean domains!} ideal, meaning that \uline{$F[x]/(p(x))$ is a field containing an isomorphic copy of $F$}.
\addlink{$F[x]/(p(x))$ field if $p(x)$ irreducible}

\defn Suppose $F\subseteq K$ are fields. Then $K$ is called a \textbf{field extension} of $F$.
\addlink{Field extension, degree of extension}
\begin{itemize}
    \item Notation: $K/F$\sidenote{Please, this is NOT a quotient. DO NOT CONFUSE THOSE!!} or \begin{tikzcd}[ampersand replacement=\&,cramped,sep=tiny]
        K \\
        F
        \arrow[no head, from=1-1, to=2-1]
    \end{tikzcd} (\textit{the lattice notation})
\end{itemize}
The dimension of $K$ as a vector space over $F$ is called the \textbf{degree} of the extension.
\begin{itemize}
    \item Notation: $[K:F]$
\end{itemize}

But does my field $F$ always have an extension? Here is a systematic way to get extensions:

\eg If $p(x)\in F[x]$ is an irreducible polynomial of degree $n\geq 1$ over the field $F$, then $F[x]/(p(x))$ is a \textbf{field extension} of $F$\sidenote{Since $\homo{F}\cong F$, and $\homo{F}\subseteq F[x]/(p(x))$} of degree $n$.\\ Furthermore, if $p(x)=a_0+a_1x+\dots +a_nx^n$, then $\bar x$ is a \textbf{root} of $$\homo{p(x)}=\bar a_0+\bar a_1y + \dots + \bar a_ny^n\in \left(\,F[x]/(p(x))\,\right)[y]$$\sidenote{We think about modding out by $(p(x))$ as making it equal to zero, which is how we find roots.} because, plugging in $y=\bar x$, we get \begin{equation*}
    \bar a_0+\bar a_1\bar x + \dots + \bar a_n\bar x^n= \overline{p(x)}=\bar 0 \in F[x]/(p(x))
\end{equation*}
Hence, \uline{the isomorphic copy of the polynomial $p(x)$ has \textbf{roots} in the field extension $F[x]/(p(x))$}.

So, what the hell is $F[x]/(p(x))$? We have already shown that the field extension $F[x]/(p(x))$ does indeed contain a root of $p(x)$. Now we think about it \textbf{the other way around}: if we want to find an extension of $F$ that contains a root of $p(x)$, we would eventually get this one!

Suppose $p(x)\in F[x]$ is irreducible. Let $K/F$ be an extension, and $\alpha\in K$ a root of $p(x)$. Denote by $F(\alpha)\subseteq K$ the \textbf{smallest} subfield of $K$ that contains both $F$ and $\alpha$. Consider the map $\varphi : F[x]\to F(\alpha)\subseteq K$ where $q(x)\mapsto q(\alpha)$ is simply the evaluation at $\alpha$ map. We note that $p(x)\in \ker(\varphi) = (d(x))$ since an ED is a PID; this implies that $p(x)=u(x)d(x)$. As $p(x)$ is irreducible, $u(x)$ must be a unit, which means $p(x)$ and $d(x)$ are associates and $\ker(\varphi)=(p(x))$. Therefore, $$F[x]/(p(x))= F[x]/\ker(\varphi)\cong \homo{F[x]}\subseteq F(\alpha)$$ by first isomorphism theorem\sidenote{Observe that $\homo{F[x]}$ is a field: $\ker(\varphi)$ is a maximal ideal}. However, $F(\alpha)\subseteq K$ the \textbf{smallest} subfield of $K$ that contains both $F$ and $\alpha$, so $\homo{F[x]}$ cannot be smaller than that. Hence, it must be true that $\homo{F[x]}=F(\alpha)$.

Therefore, \uline{$F(\alpha)$ is simply $F[x]/(p(x))$.} \qed

\subsubsection{To summarize so far!}
Suppose $p(x)\in F[x]$ is an irreducible polynomial with coefficients in the field $F$.
\begin{itemize}
    \item $F[x]/p(x)$ is a \textbf{field} containing an isomorphic copy of $F$ in which $\overline x = x+(p(x))$ is a \textbf{root} of (the image of) $p(y)\in (F[x]/(p(x)))[y]$. 
    
    \eg In $\Q[x]/(x^2-2)$, we have $x+(x^2-2)$ is a root of $y^2-\overline 2\in (\Q[x]/(x^2-2))[y]$ because \begin{align*}
        &\;(x+(x^2-2))^2 - (2+(x^2-2)) \\
        &= x^2-2 + (x^2-2) && \text{by coset addition \& multiplication}\\
        &= 0 + (x^2-2) && \text{since }x^2-2\in (x^2-2)\\
        &= \bar 0
    \end{align*}

    \item Furthermore, if $\deg(p(x))=n$, then $$F[x]/(p(x))=\left \{\,\overline{a_0+a_1x+\dots +a_{n-1}x^{n-1}}\bigm\vert a_0,a_1,\dots,a_{n-1}\in F \right \}$$
    is a vector space over $F$ of dimension $n$.
    \eg $\Q[x]/(x^2-2)=\{\bar a_0 + \bar a_1 \bar x \mid a_0, a_1\in \Q\}=\mathrm{Span}_{\Q} \{\bar 1, \bar x\}$
\end{itemize}
\begin{itemize}
    \item If $K/F$ is an extension and $\alpha\in K$ is a root of $p(x)$, denote by $F(\alpha)$\sidenote{Read `$F$ adjoint $\alpha$'} \textbf{the smallest field containing $F$ and $\alpha$}. 
    \begin{figure}[H]
        \centering
        \begin{tikzcd}[ampersand replacement=\&,cramped,sep=tiny]
            K \\
            {F(\alpha)} \\
            F
            \arrow[no head, from=1-1, to=2-1]
            \arrow[no head, from=2-1, to=3-1]
        \end{tikzcd}
        \caption{Field diagram}
        % \label{<label>}
    \end{figure}

    Then $F(\alpha)\cong F[x]/(p(x))$,\sidenote{The eval map $\varphi:F[x]\to F(\alpha)$ where $f(x)\mapsto f(\alpha)$ has in fact $\ker(\varphi)=(p(x))$ when $\alpha$ is a root of $p(x)$.} and \begin{align*}
        F(\alpha)&=\left \{\,\overline{a_0+a_1x+\dots +a_{n-1}x^{n-1}}\bigm\vert a_0,a_1,\dots,a_{n-1}\in F \right \}\\
        &= F[\alpha] \qquad \leftarrow\text{the polynomial of $\alpha$ over $F$}
    \end{align*}

    \eg $\Q(\sqrt{2}) = \{a_0+a_1\sqrt{2}\mid a_0,a_1\in \Q\} = \Q[\sqrt{2}]$
\end{itemize}

    \subsubsection{Irreducibility -- a survey}
    \begin{proposition}
        If $p(x)\in F[x]$, then $\alpha\in F$ is a root \ifnif $x-\alpha$ divides $p(x)$.
    \end{proposition}
    \begin{proof}
        Write $p(x)=(x-\alpha)q(x)+r(x)$ with $q(x),r(x)\in F[x]$ and $\deg (r(x))=0$ or $r(x)=0$. Then $0=p(\alpha)=0+r(\alpha)$ which forces $r(x)=0$.
    \end{proof}

    \begin{corollary}
        A degree-2 or -3 polynomial over a field $F$ is irreducible \ifnif it has no roots in $F$.
    \end{corollary}

    \begin{proposition}
        Suppose $p(x)=a_0+a_1x+\dots + a_nx^n\in \Z[x]$ with root $\frac{c}{d}$ written in reduced from (i.e. gcd$(c,d)=1$). Then \circled{$c|a_0$ and $d|a_n$}.
    \end{proposition}
    \begin{proof}
        \begin{align*}
            d^n \cdot p\left(\frac{c}{d}\right) &=0\\
            0&= (a_0d^n + a_1 d^{n-1} c+ \dots +a_{n-1}dc^{n-1}) + a_nc^n\\
            0&= a_0d^n + (a_1 d^{n-1} c+ \dots +a_{n-1}dc^{n-1} + a_nc^n)
        \end{align*}

        Looking at the 2nd line, since $d$ divides all of the ones in the (), it must also divide the last term $a_nc^n$. However, since gcd$(c,d)=1$, it forces $d$ to divide $a_n$.

        Similarly, we make the same argument for $c$ and $a_0$ using the 3rd line.
    \end{proof}

    \begin{lemma}
        $(R/I)[x]\cong R[x]/(I)$ where $(I)=I[x]$.
    \end{lemma}
    \begin{proof}
        Consider the surjective homomorphism $\pi: R[x]\to (R/I)[x]$.
    \end{proof}

    \begin{proposition}[Eisenstein's Criterion]
        \addlink{Eisenstein's Criterion}
        Suppose $f(x)=\mathbb1x^n+a_{n-1}x^{n-1}+\dots + a_1x+a_0\in \Z[x]$ is a monic polynomial and $p\in \Z$ is a \textbf{prime} such that $p\mid a_0,\dots,a_{n-1}$ but $p^2\nmid a_0$. Then \underline{$f(x)$ is irreducible}.
    \end{proposition}
    \begin{proof}
        Assume BWOC that $f(x)=a(x)b(x)$ for some nonunit $a(x),b(x)\in \Z[x]$, then $$x^n=\bar f(x)=\bar a(x) \bar b(x)$$
        in $(\Z/p\Z)[x]\cong \Z[x]/p\Z[x]$ since all other terms are divisible by $p$. Since $\Z/p\Z$ does not contain any zero divisors, $\bar a(x), \bar b(x)$ must have zero constant terms. Hence $a(x),b(x)$ have constant terms that are multiples of $p$, so $a(x)b(x)$ have constant term divisible by $p^2$. This is a contradiction with $p^2\nmid a_0$.
    \end{proof}

    \begin{lemma}[Gauss' Lemma]
        \addlink{Gauss' Lemma}
        If $p(x)\in \Z[x]$ is reducible in $\Q[x]$, then it is reducible in $\Z[x]$.
    \end{lemma}
    \begin{proof}
        Suppose $p(x)=a(x)b(x)$ for $a(x),b(x)\in \Q[x]$. Then by multiplying by coefficient denominators, for some $m\in \Z$, we could write $m\cdot p(x)= c(x)d(x)$ for some $c(x),d(x)\in \Z[x]$. Now since $m\in \Z$, we could write $m=q_1q_2\dots q_n$ be a product of irreducibles in $\Z$.
        
        Now in $(\Z/q_1\Z)[x]\cong \Z[x]/(q_1\Z)[x]$, we observe that $m\cdot p(x)= c(x)d(x)=q_1(q_2\dots q_n)p(x)$, meaning that \begin{align*}
            \overline{c(x)}\,\overline{d(x)}=\overline{q_1(q_2\dots q_n)p(x)}=\overline 0
        \end{align*}
        Since $(\Z/q_1\Z)[x]\cong \Z[x]/(q_1\Z)[x]$ is an \uline{integral domain}\sidenote{since $q_1$ is irreducible and hence prime in UFD}, WLOG, $\overline{c(x)}=\overline 0$ \ifnif $c(x)\in q_1\Z[x]$, meaning that all coefficients of $c(x)$ are multiples or $q_1$. Therefore, $\dfrac{1}{q_1}c(x)\in \Z[x]$.

        Now we repeat the process for all $q_1,q_2,\dots ,q_n$ and we are done.
    \end{proof}

    Recall that if $F\subseteq K$ are fields, $\alpha\in K$ and $p(x)\in F[x]$ is irreducible with root $\alpha$, then $$F[\alpha]=F(\alpha)\cong F[x]/(p(x))=\{\overline{a_0+a_1x+\dots+a_{n-1}x^{n-1}}\mid a_0,a_1\dots,a_{n-1}\in F\}$$

    We observe that this has a few implications. For instance, $F(\alpha)$ contains $\frac{1}{\alpha}$\sidenote{since it is a field containing the mult. inverse of $\alpha$}, meaning that it could also be written as a polynomial of $\alpha$ with coefficients in $F$ (as in $F[\alpha]$)!

    \defn Suppose $K/F$ is a field extension and $\alpha\in K$. We say that $\alpha$ is \textbf{algebraic over $F$} if there exists $p(x)\in F[x]$ such that $p(\alpha)=0$. If not, $\alpha$ is \textbf{transcendental}.
    \addlink{Algebraic and transcendental}

    \defn The extension $K/F$ is an \textbf{algebraic extension} if \textbf{every} element $\alpha\in K$ is algebraic over $F$.
    \addlink{Algebraic extension}

    \eg $\pi$ is transcendental over $\Q$ but algebraic over $\R$ (since it is a root of $x-\pi$).

    \begin{proposition}
        If $K/F$ is a \textbf{finite extension} \sidenote{finite extension just means finite degree $[K:F]<\infty$}, then it is an algebraic extension.
    \end{proposition}
    \addlink{Finite extensions are algebraic}
    \begin{proof}
        Call $[K:F]=n$ and let $\alpha\in K$. Then the $n+1$ elements $\{1,\alpha,\alpha^2,\dots, \alpha^n\}$ must be \uline{linearly dependent}\sidenote{since $n+1>\dim(K/F)=n$}. Hence, by linear algebra, there exist $a_0,a_1,\dots,a_n\in F$ not all zero such that the linear combination $a_0+a_1\alpha+a_2\alpha^2+\dots +a_n\alpha^n=0$. Hence, $\alpha$ is a root of $a_0+a_1x+a_2x^2+\dots +a_nx^n=0$.
    \end{proof}

    \begin{corollary}
        If $K/F$ is an extension and $\alpha\in K$, then $\alpha$ is algebraic over $F$ \ifnif $[F(\alpha):F]<\infty$.
    \end{corollary}
    \begin{proof}\hfill
        \begin{enumerate}[align=left]
            \item[($\impliedby$)] Follows from prop.
            \item[($\implies$)] If $\alpha$ is algebraic, then there exists an irreducible polynomial $p(x)$ with $\alpha$ as a root and of degree $n<\infty$. Then $F(\alpha)\cong F[x]/(p(x))$ is a $n$-dimensional vector space over $F$.
            
            Another perspective: $F(\alpha)=\mathrm{Span}_F\{1,\alpha,\alpha^2,\dots ,\alpha^{n-1}\}$.\sidenote{Review proof of $F(\alpha)\cong F[x]/(p(x))$.}
        \end{enumerate}
    \end{proof}

    \begin{proposition}
        Suppose $K/F$ is an extension \& $\alpha\in K$ is algebraic over $F$. Then there exists a \uline{unique, irreducible, and monic} polynomial $m_{\alpha,F}(x)\in F[x]$ that has $\alpha$ as a root.
    \end{proposition}
    \addlink{Minimal polynomial}
    \rmk We observe that $m$ does depend on the base field $F$; $m_{\sqrt{2},\Q}(x)=x^2-2$, but $m_{\sqrt{2},\Q(\sqrt{2})}(x)=x-\sqrt{2}$.
    \begin{proof}
        Since the subset of $F[x]$ \textit{satisfying $\alpha$ is a root} is nonempty, we can pick one with a \textbf{minimal degree}. By multiplying by an element of $F$ if necessary, we can assume WLOG this polynomial is \textbf{monic}. Call it $m_{\alpha,F}(x)$. 
        
        Assume BWOC that $m$ is the product of two other polynomials of lesser degree such that $m_{\alpha,F}(x)=a(x)b(x)$, then we plug in $0=m_{\alpha,F}(\alpha)=a(\alpha)b(\alpha)$. Since there are no zero divisors in $F[x]$, WLOG $a(\alpha)=0$, contradicting the minimality of $m_{\alpha,F}$. Hence $m_{\alpha,F}$ is \textbf{irreducible}.\sidenote{So $F(\alpha)\cong F[x]/(m_{\alpha,F}(x))$}

        Then, BWOC if $p(x)\in F[x]$ with $\alpha$ as a root and is monic and irreducible, there exist $q(x),r(x)\in F[x]$ such that $p(x)=m_{\alpha,F}(x)q(x)+r(x)$ where $\deg(r)< \deg(m_{\alpha,F})$ or $r(x)=0$. Then, we observe that $p(\alpha)=0=m_{\alpha,F}(\alpha)q(m_{\alpha,F})+r(\alpha)=0+r(\alpha)$. Thus, $r(\alpha)=0$, so $\deg(r)\geq \deg(m_{\alpha,F})$ unless $r(x)=0$ by minimality. Hence we must have $r(x)=0$, so $m_{\alpha,F}|p$. This contradicts the assumption that $p$ is monic and irreducible.
        Therefore, $m_{\alpha,F}$ is the \textbf{only} minimal, monic and irreducible polynomial where $\alpha$ is a root.
    \end{proof}

    \defn $m_{\alpha, F}(x)$ is the \textbf{minimal} polynomial of $\alpha$ over $F$.

    \ontangent{
        
    Some exam prep!
    \begin{itemize}
        \item In general, for subrings $R\subseteq S$, we have if $r\in R^{\times}$, then $r\in S^{\times}$.
        \item If we adjoint one root of an irreducible polynomial to a field, the fields are isomorphic no matter which root of that polynomial we adjoint.
    \end{itemize}}

To summarize, if $K/F$ is a field extension and $\alpha\in K$, then $\alpha$ is \textbf{algebraic} over $F$ if it is the root of some polynomials in $F[x]$. For each algebraic $\alpha$, there exists a unique, monic, irreducible polynomial $m_{\alpha,F}(x)\in F[x]$ such that $m(\alpha)=0$. In that case, the degree of extension $[F(\alpha):F]=\deg(m_{\alpha,F}(x))$; and, if $p(\alpha)=0$ for some $p(x)\in F[x]$, then $m_{\alpha,F}\vert p(x)$. In general, if $[K:F]<\infty$, then $K/F$ is algebraic. Thus, $[F(\alpha):F]<\infty$ \ifnif $\alpha$ is algebraic over $F$.
\begin{proposition} 
    If $F\subseteq K\subseteq L$ are fields, then \sidenote{\begin{tikzcd}[ampersand replacement=\&,cramped,sep=small]
        L \\
        K \\
        F
        \arrow["n", no head, from=1-1, to=2-1]
        \arrow["m", no head, from=2-1, to=3-1]
        \arrow["mn"', no head, bend right = 30, from=1-1, to=3-1]
    \end{tikzcd}}\begin{align*}
        [L:F]=[L:K]\cdot[K:F]
    \end{align*}
    \addlink{Degree of extension is multiplicative}
\end{proposition}

\begin{proof}
    We first see that if $[K:F]=\infty$, then for any $N\in \N$, there exists $\alpha_1,\dots,\alpha_N\in K$ that are linearly independent over $F$. In that case, it is certainly true that $\alpha_1,\dots,\alpha_N\in L$ are linearly independent over $F$. Thus, $[L:F]=\infty$. \sidenote{Linear independence implies that whenever $a_1\alpha_1+a_2\alpha_2+\dots+a_N\alpha_N=0$ for some coefficients $a_1,\dots,a_N\in F$, then necessarily $a_1=a_2=\dots=a_N=0$.}

    If $[L:K]=\infty$, then for any $N\in \N$, there exists $\beta_1,\dots,\beta_N\in L$ that are linearly independent over $K$. As a result, it also is linearly independent over $F$. Hence, $[L:F]=\infty$.

    If $[K:F]=m$ and $[L:K]=n$, let $\alpha_1,\dots,\alpha_m\in K$ be a basis for $K$ over $F$ and $\beta_1,\dots,\beta_n\in L$ be a basis for $L$ over $K$. 
    
    \rt{Claim:} $\{\alpha_i\beta_j\mid 1\leq i\leq m, 1\leq j\leq n\}$ forms a basis for $L$ over $F$.\sidenote{Use linear combinations to prove this claim.}
\end{proof}

Some nice consequences:
\begin{corollary}
    Suppose $K/F$ is an extension and $\alpha,\beta\in K$ are algebraic over $F$. Then:
    \begin{itemize}
        \item $F(\alpha,\beta)=(F(\alpha)(\beta))$\sidenote{the smallest subfield of $K$ containing $F, \alpha,\beta$}
        \item $[F(\alpha,\beta):F]=[F(\alpha,\beta):F(\beta)][F(\beta):F]=\deg\left(m_{\alpha,F(\beta)}(x)\right)\cdot \deg\left(m_{\beta,F(x)}\right)$. However, note that the minimal polynomial \sidenote{since $p(\alpha)=0\iff m_{\alpha,F}(x)|p(x)$}$$m_{\alpha,F(\beta)}(x)\mid m_{\alpha,F}(x)\in F(\beta)[x]$$
        so $\deg(m_{\alpha,F(\beta)}(x))\leq \deg(m_{\alpha,F}(x))$. Hence, $$[F(\alpha,\beta):F]\leq \deg(m_{\alpha,F}(x))\deg(m_{\beta,F})<\infty$$
        This means that whenever $\alpha, \beta$ are algebraic over $F$, we get that $F(\alpha,\beta)/F$ is an algebraic extension.
        \addlink{Algebraic elements form a field}
        \item As a result, $\alpha\pm\beta, \alpha\beta, \alpha/\beta$ are all algebraic over $F$. The algebraic elements hence form a \textbf{field}.
    \end{itemize}
\end{corollary}

\addlink{Extension finite iff. adjoin finite algebraic elements}
\begin{proposition}
    Suppose $K/F$ is an extension. Then $[K:F]< \infty$ \ifnif $K=F(\alpha_1,\dots,\alpha_n)$ could be written where $\alpha_1,\dots,\alpha_n\in K$ are algebraic over $F$. 
    
    In other words, an extension is finite \ifnif it is generated by adjoining a finite amount of algebraic elements.
\end{proposition}
\begin{proof}\hfill
    \begin{itemize}[align=left]
        \item[($\implies$)] If $[K:F]< \infty$, then suppose $\{\alpha_1,\dots,\alpha_n\}$ is a basis of $K$ over $F$. Then $\alpha_1,\dots,\alpha_n$ are algebraic and every element of $K$ is an $F$-linear combination of $\alpha_i$s. Hence $K$ must be the smallest field containing $F$ and $\alpha_i$s, which means $K=F(\alpha_1,\dots,\alpha_n)$.
        \item[($\impliedby$)] We observe that \begin{align*}
            [K:F] &= [(F(\alpha_1,\dots,\alpha_{n-1}))(\alpha_n):F(\alpha_1,\dots,\alpha_{n-1})]\cdot \ldots\cdot [F(\alpha_1):F]\\
            &\leq \prod_{i=1}^{n}\deg(m_{\alpha_i,F}(x))<\infty
        \end{align*}
    \end{itemize}
\end{proof}

\begin{corollary}
    If $L/K$ and $K/F$ are algebraic extensions, then so is $L/F$.\sidenote{$L/K$ and $K/F$ need not be finite!}
\end{corollary}
\begin{proof}
    Suppose $\alpha\in L$. Since $L/K$ is algebraic, there exists $p(x)\in K[x]$ such that $p(\alpha)=0$. Let $\alpha_0,\dots,\alpha_n\in K$ be the coefficients of $p(x)$, necessarily algebraic over $F$ since $K/F$ algebraic. Therefore, $$[F(a_0,\dots,a_n,\alpha):F]=[(F(a_0,\dots,a_n))(\alpha):F(a_0,\dots,a_n)][F(a_0,\dots,a_n):F]$$
    Since $p(\alpha)=0$ has coefficients in $K\supseteq F(\alpha_0,\dots,\alpha_n)$, we have $[(F(a_0,\dots,a_n))(\alpha):F(a_0,\dots,a_n)]<\infty$. The second term is also clearly $<\infty$. Therefore, $[F(a_0,\dots,a_n,\alpha):F]<\infty$, meaning that $\alpha$ is algebraic over $F$.\sidenote{\begin{tikzcd}[ampersand replacement=\&,cramped, sep=tiny]
        L \\
        \& {(F(a_0,\dots,a_n))(\alpha)} \\
        K \\
        \& {F(a_0,\dots,a_n)} \\
        F
        \arrow[no head, from=1-1, to=2-2]
        \arrow[no head, from=1-1, to=3-1]
        \arrow[no head, from=3-1, to=5-1]
        \arrow[no head, from=3-1, to=4-2]
        \arrow[no head, from=2-2, to=4-2]
        \arrow[no head, from=4-2, to=5-1]
    \end{tikzcd}}
\end{proof}

\defn Suppose $L/F$ is an extension \& $K_1$ and $K_2$ are intermediate fields. The \textbf{composite} field $K_1K_2$ is the smallest subfield of $L$ containing $K_1$ and $K_2$.\addlink{Composite field}
\[\begin{tikzcd}[ampersand replacement=\&,cramped,sep=tiny]
	\& L \\
	\& {K_1K_2} \\
	{K_1} \&\& {K_2} \\
	\& {K_1\cap K_2} \\
	\& F
	\arrow[no head, from=4-2, to=5-2]
	\arrow[no head, from=3-1, to=4-2]
	\arrow[no head, from=3-3, to=4-2]
	\arrow[no head, from=2-2, to=3-1]
	\arrow[no head, from=2-2, to=3-3]
	\arrow[no head, from=1-2, to=2-2]
\end{tikzcd}\]

\defn \addlink{Splitting field}Suppose $F$ is a field and $p(x)\in F[x]$. The \textbf{splitting field} of $p(x)$ over $F$ is the smallest field extension of $F$ over which $p(x)$ could be factored into \textbf{linear factors}.\sidenote{Assuming that splitting fields \textbf{exist} and are \textbf{unique} up to isomorphism.}

\rmk If $E$ is the splitting field of $p(x)$ over $F$ then $[E:F]\leq n!$ where $n=\deg(p(x))$.

\rmk Such an extension is called \textbf{normal}.
\addlink{Normal extensions}

\begin{proposition}
    \addlink{Splitting fields exist}
    Splitting fields exist.
\end{proposition}
\begin{proof}[Proof outline]
    By induction on $\deg(p(x))$, whose base case, $\deg(p(x))=1$, yields $F$ as a splitting field. More generally, any $p(x)$ has a root $\alpha$ in $F(\alpha)\cong F[x]/(q(x))$ for some irreducible $q(x)$ so $p(x)=(x-\alpha)f(x)\in F(\alpha)[x]$. We observe that $\deg(f(x))=\deg(p(x))-1$. Induction takes care of the rest.\sidenote{The splitting field of $p(x)$ over $F$ is the same as the splitting field of $f(x)$ over $F(\alpha)$}
\end{proof}

\rmk $K$ is a splitting field over $F$ \ifnif every irreducible $p(x)\in F[x]$ that has one root in $K$ has \textbf{all} its roots in $K$.
\noneg $\Q(\sqrt[3]{2})$ over $\Q$ is not such an extension.

\begin{lemma}\label[lemma]{isom-between-field-extensions}
    Suppose $\varphi: F_1\to F_2$ is a field isomorphism, $p_1(x)\in F_1[x]$, and $p_2(x)=``\homo{(p_1(x))}''$ ($\varphi$ applied to coeffs of $p_1(x)$).\sidenote{In this way, $\varphi$ induce a ring isomorphism $F_1[x]\to F_2[x]$.} Let $\alpha_1$ be a root of the irreducible factor $q_1(x)$ of $p_1(x)$, and let $q_2(x)=``\homo{q_1(x)}''$ and $\alpha_2$ be a root of $q_2(x)$. Then there exists an isomorphism $\tau: F_1(\alpha_1)\to F_2(\alpha_2)$ such that $\tau(\alpha_1)=\alpha_2$ and $\tau_{|F_1}=\varphi$ (this means ``$\tau$ restricted to $F_1$'').

    \begin{proof}[Proof outline]
        \[\begin{tikzcd}[ampersand replacement=\&,cramped,column sep=normal, row sep=tiny]
            \& {F_1(\alpha_1)} \& {F_1[x]/(q_1(x))} \& {F_2[x]/(q_2(x))} \& {F_2(\alpha_2)} \\
            \& {\alpha_1} \& {\bar x} \& {\bar x} \& {\alpha_2} \\
            {\text{if }a\in F_1,} \& a \& {\bar a} \& {\overline{\varphi(a)}} \& {\varphi(a)}
            \arrow["\sim", from=1-2, to=1-3]
            \arrow["\sim", from=1-3, to=1-4]
            \arrow["\sim", from=1-4, to=1-5]
            \arrow[from=2-2, to=2-3]
            \arrow[from=2-3, to=2-4]
            \arrow[from=2-4, to=2-5]
            \arrow[from=3-2, to=3-3]
            \arrow[from=3-3, to=3-4]
            \arrow[from=3-4, to=3-5]
        \end{tikzcd}\]
    \end{proof}
\end{lemma}

\begin{proposition}
    Suppose $F_1,F_2,\varphi,p_1(x)$ and $p_2(x)$ are as in \cref{isom-between-field-extensions}. Let $E_1\& E_2$ be splitting fields of $p_1$ and $p_2$ respectively. Then there exists an isomorphism $\sigma: E_1\to E_2$ such that $\sigma_{|F_1}=\varphi$.\sidenote{if we set $F_1=F_2$ and $p_1=p_2$, we get corollary: splitting fields are unique up to isomorphism.}

    \begin{proof}
        Proceed by induction on $\deg(p_1(x))$. For the base case, if $\deg(p_1(x))=1$, then $E_1=F_1$ and $\sigma=\varphi$.

        Assume the result is true for all polynomials of fixed degree $k\geq 1$ and suppose $\deg(p_1(x))=k+1$. Let $\alpha_1$ be a root of $p_1(x)$ and $\alpha_2$ be a root of the $\varphi$-corresponding irreducible factor of $p_2(x)$. By \cref{isom-between-field-extensions}, $\varphi$ can be extended to $\tau: F_1(\alpha_1)\to F_2(\alpha_2)$ such that $\tau_{|F_1}=\varphi$.
        
        In $(F_1(\alpha_1))[x]$, we can factor out $p_1(x)=(x-\alpha_1)g_1(x)$, and in $(F_2(\alpha_2))[x]$ we factor $p_2(x)=(x-\alpha_2)g_2(x)$ with $g_2(x)=\tau(g_1(x))$.\sidenote{
            \begin{tikzcd}[ampersand replacement=\&,cramped,sep=tiny]
                {\sigma:} \& {E_1} \&\& {E_2} \\
                {\tau:} \& {F_1(\alpha_1)} \&\& {F_2(\alpha_2)} \\
                {\varphi:} \& {F_1} \&\& {F_2}
                \arrow["\sim", from=1-2, to=1-4]
                \arrow["\sim", from=2-2, to=2-4]
                \arrow["\sim", from=3-2, to=3-4]
                \arrow[no head, from=2-2, to=3-2]
                \arrow[no head, from=2-4, to=3-4]
                \arrow[no head, from=1-2, to=2-2]
                \arrow[no head, from=1-4, to=2-4]
                \arrow[no head, from=1-1, to=2-1]
                \arrow[no head, from=2-1, to=3-1]
            \end{tikzcd}
        } We observe that $E_1$ and $E_2$ are the splitting fields of $g_1$ and $g_2$ over $F_1(\alpha_1)$ and $F_2(\alpha_2)$!

        By inductive hypothesis, $\tau$ could be extended to $\sigma$ and $\sigma_{|F_1(\alpha)}=\tau$ and $\sigma_{|F_1}=\varphi$.
    \end{proof}
\end{proposition}

\begin{corollary}
    Splitting fields are unique.
    \addlink{Splitting fields are unique}
    \begin{proof}
        Set $F_1=F_2$, $\varphi=$id, $p_1(x)=p_2(x)$.
    \end{proof}
\end{corollary}

\ontangent{

    Homework hint: the proof of existence \& uniqueness of splitting fields relied on inductive arguments where we adjoin one root at a time. This is the same as saying $E=F(\alpha_1,\dots,\alpha_n)$ but this tends to overlook isomorphic ways to adjoin roots. In this context, it is convenient to start by considering a specific $K$ containing $F$ and all roots of $p(x)$. In that case, $E=F(\alpha_1,\dots,\alpha_n)$ becomes more rigorous.

}

\defn A polynomial is called \textbf{separable} if it doesn't have repeated roots.\sidenote{First note that a poly of degree $n$ over a field has exactly $n$ roots.}\addlink{Separable polynomials}

\defn Suppose $f(x)=a_nx^n+a_{n-1}x^{n-1}+\dots + a_1x+a_0$. The \textbf{formal derivative} of $f(x)$ is the polynomial $$D_xf(x)=f'(x)=na_{n}x^{n-1}+(n-1)a_{n-1}x^{n-2}+\dots+a_1$$\addlink{Formal derivatives}

From this definition, we can check that the usual differential rules hold.

\begin{lemma}
    Suppose $F$ is a field, $f(x)$ is a polynomial in $F[x]$, and $E/F$ a field extension containing a root $\alpha$ of $f(x)$. Then $\alpha$ is a repeated root of $f(x)$ \ifnif $\alpha$ is a root of the formal derivatve $f'(x)$. \addlink{Roots repeated iff. roots of derivative}

    \begin{proof}
        If $\alpha$ is a repeated root of $f(x)$ then $f(x)=(x-\alpha)^2g(x)$ for some $g(x)\in E[x]$. In that case, $f'(x)=2(x-\alpha)g(x)+(x-\alpha)^2g'(x)$ and so $f'(\alpha)=0$.

        Conversely, if $f'(\alpha)=0$, then differentiating $f(x)=(x-\alpha)h(x)$ (where $h(x)\in E[x]$) and plugging $x=\alpha$ yields $0=f'(\alpha)=h(\alpha)+(\alpha-\alpha)h'(\alpha)=h(\alpha)$. This is saying that $h(x)=(x-\alpha)g(x)$ for some $g(x)\in E[x]$.
    \end{proof}
\end{lemma}

\addlink{Irreducible non-separables have $f'(x)=0$}
\begin{lemma}\label[lemma]{irred-non-sep}
    If $f(x)\in F[x]$ is\textbf{ irreducible and not separable}, then $f'(x)=0$.

    \begin{proof}
        If $f(x)$ is not separable, we know that there is at least one repeated root. We call it $\alpha$. Then let $m_{\alpha,F}(x)$ be the minimal polynomial of $\alpha$ over $F$ and we have $f(x)=c\cdot m_{\alpha,F}(x)$ for some constant $c\in F$. Therefore, $\deg(f(x))=\deg(m_{\alpha,F}(x))$. However, by the previous lemma, $f'(\alpha)=0$ must also exist since $\alpha$ is a repeated root. If $f'(x)\neq 0$, then we found a polynomial with degree less than the minimal polynomial that has $\alpha$ as a root, which cannot happen. Therefore, $f'(x)=0$.
    \end{proof}
\end{lemma}

If $f(x)$ is not constant and $f'(x)=0$, then $\mathrm{char} (F)=p>0$ and $f(x)=g(x^p)$.\sidenote{All powers in $f(x)$ are multiples of $p(x)$.}

\begin{proposition}
    If $\mathrm{char}(F)=0$, or $|F|<\infty$ and $\mathrm{char}(F)=p $, then every irreducible polynomial in $F[x]$ is separable.
\end{proposition}
\addlink{Irreducible polys are separable in finite or char 0 fields}
\begin{proof}
    For the case of char=0, it follows from \ref{irred-non-sep}.

    For the case of char>0, suppose $F$ is a finite field of $p^n$ elements\footnote{See \cref{sec:prime-fields}}. \sidenote{Use binomial theorem.}Then the map $F\to F$ where $\alpha\mapsto \alpha^p$ is a \dashuline{field isomorphism}. Hence, \dashuline{every element of $F$ is a $p^{th}$ power.}\sidenote{Such fields are called \textbf{perfect}.}

    Now suppose BWOC $f(x)=\sum_{i=0}^{n}a_ix^i\in F[x]$ is an irreducible but not separable polynomial. Therefore, $f'(x)=0$ must be true. This happens \ifnif $f(x)=\sum_{j=0}^{m}a_{jp}x^{jp}$, that is, the $x$ in all terms are of $p^{th}$ degree. However, we know that all elements $a_{jp}\in F$ are already the $p^{th}$ powers of sth else $(b_{jp})^p=a_{jp}$, so $$f(x)=\sum_{j=0}^{m}(b_{jp}^p)x^{jp}$$ and by reverse Binomial Theorem, we get $$f(x)=\sum_{j=0}^{m}(b_{jp}^p)x^{jp}=\left(\sum_{j=0}^{m}b_{jp}x^j\right)^p$$ is not irreducible!
\end{proof}
\noneg{Let $F=\F_p(t)=\left \{\frac{f(t)}{p(t)} \biggm | f(t),g(t)\in \F_p[t],g(t)\neq 0\right \}$.\sidenote{This is a field of char>0 but is infinite.} 

Then $p(x)=x^p-t$ is not separable (but it is irreducible). \sidenote{The coefficients of $p(x)$ are ratios of polys in $\F_p(t)$.}
This can be seen if we suppose $\alpha$ is a root of $p(x)$ (so $\alpha=t$). Then, in $F(\alpha)[x]$, we have $p(x)=x^p-t=x^p-\alpha^p=(x-\alpha)^p$, which tells us $p(x)$ is not separable.}

\ontangent{
    \subsubsection{Prime fields}\label[section]{sec:prime-fields} Suppose $R$ is a commutative ring with identity. The map $\Z\to R$ where $n\mapsto \pm(\underset{|n| \text{ times}}{\underbrace{1_R+1_R+\dots +1_R}})$ ($-$ if $n<0$) is a \uline{ring homomorphism}\sidenote{check it!} with kernel $n\Z$ where $n=\mathrm{char}(R)$. So:\begin{itemize}
        \item if char $(R)=0$, then $R$ contains $\Z$;
        \item if char $(R)=n>0$, then $R$ contains $\znz{\textit{n}}$.
    \end{itemize}
    If $F$ is a field, then:\begin{itemize}
        \item if char $(F)=0$, then $F$ conatins $\Q$;
        \item if char $(F)=p>0$, then $p$ prime and $F$ contains $\znz{\textit{p}}$.\sidenote{That is, $F$ is an extension of $\znz{\textit{p}}$!}
    \end{itemize}
    \addlink{Every field is an extension of $\Q$ or $\F_p$}
    In other words, \dashuline{every field is an extension of $\Q$ or $\F_p$.} Moreover, a finite field is a \textit{finite} extension of $\F_p$: if $[F:\F_p]=n$, then $|F|=p^n$.\sidenote{All elts of $F$ are $a_0+a_1x_1+\dots+a_nx_n$ where $a_i\in \F_p$, so we have $p^n$ choices.}

    In addition, $|F-\{0\}|=p^n-1\implies $ if $\alpha\in F-\{0\}$ then $\alpha^{p^n-1}=1$\sidenote{By Lagrange's Theorem}, implying that if $\alpha\in F$, then $\alpha^{p^n}=\alpha$, meaning that $\alpha$ is a root of $x^{p^n}-x\in F[x]$. Therefore, $F$ is the \dashuline{splitting field} of $x^{p^n}-x$. But splitting fields are unique, so we conclude that there is only one unique finite field for each order.
    \addlink{Finite fields are unique}

}

\addlink{Separable extensions}
\defn An algebraic extension $K/F$ is called (algebraically) \textbf{separable} if $m_{\alpha,F}(x)$ is separable for \textbf{all} $\alpha\in K$.

\section{Galois Theory}

\addlink{Galois extensions}
\defn A finite extension $K/F$ is called \textbf{Galois} if $K/F$ is \uline{normal}\sidenote{normal just means it is a splitting field of something} and \uline{separable}.


\defn \label[definition]{auto-group}If $K/F$ is an extension, then the \textbf{automorphism group} of $K/F$ is defined as \begin{align*}
    \Aut(K/F) &= \left\{ \sigma\in \Aut(K) \mid \sigma(a)=a\, \forall \, a\in F\right\}
\end{align*}\addlink{Automorphism group of field extensions}
That is, all the automorphisms of $K$ that also fix the field $F$.

\spl
\addcontentsline{toc}{subsection}{Phase III: Galois Theory}
\begin{tcolorbox}
    \textbf{Galois theory} is concerned with the \dashuline{study of roots of polynomials} by way of \dashuline{automorphisms of splitting fields} (of separable polynomials).

    In particular, we are interested in what $$\Aut(K)=\{\sigma:K\to K \text{ isomorphisms}\}$$ (a group under composition) is. Naturally, such groups are \textit{finite}.
\end{tcolorbox}\sidenote{review MATH171 finite groups!}
\spl

Last time, we showed that $K\supseteq \left\{\begin{matrix}
    \Q & \text{ if char}(K)=0\\
    \F_p &\text{ if char}(K)=p
\end{matrix}\right\}$, since $\sigma(1)=\sigma(1^2)=(\sigma(1))^2$ implies that $\sigma(1)=1$ must always be true! Hence, $\sigma(n)=n$ must be true in char 0 fields, or $\sigma(\bar n)=\bar n$ if char>0 for all $n\in \Z$. Therefore, $\sigma$ \textbf{fixes} the prime subfield $\Q$ or $\F_p$.

\rmk Why does \cref{auto-group} have to fix the field $F$? Because we've shown that $\Aut(K)=\left\{\begin{matrix}
    \Aut(K/\Q) & \text{ if char}(K)=0\\
    \Aut(K/\F_p) &\text{ if char}(K)=p
\end{matrix}\right\}$.

\begin{lemma}
    If $K/F$ is an extension, $\alpha\in K$ is algebraic over $F$ and $\sigma\in \Aut(K/F)$, then $\sigma(\alpha)$ is a root of $m_{\alpha,F}(x)$.
\end{lemma}
\addlink{Roots of minimal polys under automorphisms are still roots}
\begin{proof}
    Observe that $m_{\alpha,F}(\alpha)=0=\sigma(0)=\sigma(m_{\alpha,F}(\alpha))$. Hence, since $\sigma(\alpha)=\alpha$ for all $\alpha\in F$, $$\sigma(a_0+a_1\alpha+\dots +a_n\alpha^n)=a_0+a_1\sigma(\alpha)+\dots + a_n\sigma(\alpha)^n=m_{\alpha,F}(\sigma(\alpha))$$
\end{proof}

So\sidenote{Recall: if irreducible $f(x)$ has one root in such $K$, then all roots lie in $K$.} if $f(x)\in F[x]$, then every $\sigma\in \Aut(K/F)$ \textbf{permutes} the roots of $f(x)$ that lie in $K$. It would be nice if the roots of $f(x)$ all lived in $K$. This is why we consider $K$ the splitting field of some polynomial over $F$.

\begin{proposition}
    If $K$ is the splitting field of some polynomial $f(x)$ over $F$ (so $[K:F]<\infty$), then $|\Aut(K/F)|\leq [K:F]$, with equality if $f(x)$ is separable.
\end{proposition}
\begin{proof}
    We will prove a more general statement by induction.\sidenote{The prop above fixes $F_1=F_2$ and $\sigma$ being identity.}
    If $\sigma:F_1\to F_2$ is an isomorphism, $f_1(x)\in F_1[x]$ and $f_2(x)=\sigma(f_1(x))\in F_2[x]$ and $E_1$ and $E_2$ are the splitting fields of $f_1$ and $f_2$ over $F_1$ and $F_2$ respectively. Then we would like to show that there are at most $[E_1:F_1]$ isomorphisms $\tau: E_1\to E_2$ such that $\tau_{|F_1}=\sigma$ with equality if $f_1$ separable.

    \begin{enumerate}[align=left]
        \item[\textit{Base case. }] If $[E_1:F_1]=[E_2:F_2]=1$, then $E_1=F_1$, $E_2=F_2$ and $\tau=\sigma$ is our only choice.
        
        \sidenote{\begin{tikzcd}[ampersand replacement=\&,cramped,sep=small]
            {\tau:} \& {E_1} \& {E_2} \\
            \& \vdots \& \vdots \\
            {\rho:} \& {F_1(\alpha)} \& {F_2(\beta)} \\
            {\sigma:} \& {F_1} \& {F_2}
            \arrow["\sim", from=4-2, to=4-3]
            \arrow["{> 1}"', no head, from=3-2, to=4-2]
            \arrow["\sim", from=3-2, to=3-3]
            \arrow[no head, from=3-3, to=4-3]
            \arrow[no head, from=2-2, to=3-2]
            \arrow[no head, from=2-3, to=3-3]
            \arrow[no head, from=1-3, to=2-3]
            \arrow["\sim", from=1-2, to=1-3]
            \arrow[no head, from=1-2, to=2-2]
            \arrow["{< n}"', draw=none, from=1-2, to=3-2]
        \end{tikzcd}}
        \item[\textit{Inductive step. }] Suppose we've proven the result for all extensions of degree $<n$ for some $n\geq 2$. Now consider $[E_1:F_1]=[E_2:F_2]=n$. Pick $\alpha\in E_1\backslash F_1$ and let $\beta\in E_2$ be any root of $\sigma(m_{\alpha,F_1}(x))$. Then $\sigma$ could be extended to $\rho:F_1(\alpha)\to F_2(\beta)$ such that $\rho(\alpha)=\beta$ and $\rho_{|F_1}=\sigma$. Observe that $[F_1(\alpha):F_1]=\deg(m_{\alpha,F_1}(x))$. Moreover, the number of extensions of $\sigma$ to $\rho$ equals the number of\textit{ distinct} roots of $\sigma(m_{\alpha,F_1}(x))$. Thus, the number of extensions of $\sigma$ to $F_1(\alpha)$ is \textit{at most} the degree of $m_{\alpha,F_1}(x)$ which is $[F_1(\alpha):F_1]$ with equality if $m_{\alpha,F_1}(x)$ is separable. Since $[E_1:F_1]=[E_2:F_2]=n$, we have $[E_1:F_1(\alpha)]<n$, by inductive hypothesis, there are at most $[E_1:F_1(\alpha)]$ ways of extending $\rho$ to $\tau:E_1\to E_2$. Hence, \begin{align*}
            |\left\{ \text{extensions of }\sigma \text{ to } \tau\right\}| &= |\left\{ \text{extensions of }\sigma \text{ to } \rho\right\}||\left\{ \text{extensions of }\rho \text{ to } \tau\right\}|\\
&\leq [F_1(\alpha):F_1][E_1:F_1(\alpha)]\\
&=[E_1:F_1]
        \end{align*}
    \end{enumerate}
    Looking at the case $F_1=F_2$, $E_1=E_2$, $\sigma=\mathrm{id}$, we get our result.
\end{proof}

\defn If $K/F$ is a normal extension, then the extension is \textbf{Galois} if $[K:F]=|\Aut(K/F)|$.
\addlink{Galois extensions (alternate def)}

\rmk Notation: if $K/F$ is Galois, then use $\Gal(K/F)$ for $\Aut(K/F)$.

\subsection{Fixed Fields and Automorphism Groups}
\defn Suppose $K/F$ is a field extension. If subgroup $H\leq \Aut(K/F)$, then the \textbf{fixed field} of $H$ is given by $K_H = \{\alpha\in K \mid \sigma(\alpha)=\alpha \text{ for all } \sigma\in H\}$. \sidenote{Observe that $K_H$ is indeed a field (the sum, products etc. are also fixed by $\sigma$); moreover, it is an intermediate extension $F\subseteq K_H \subseteq K$.}
\addlink{Fixed field}

\rmk Also, observe that if $F\subseteq E\subseteq K$, then $\Aut(K/E)\leq \Aut(K/F)$.

\begin{lemma}
    Suppose $K/F$ is an extension. Then: \begin{enumerate}[label=(\arabic{*})]
        \item If $H_1,H_2\leq \Aut(K/F)$ with $H_1\leq H_2$, then $K_{H_2}\subseteq K_{H_1}$.
        \item If $F\subseteq E_1\subseteq E_2\subseteq K$ are two intermediate extensions, then $\Aut(K/E_2)\leq \Aut(K/E_1) \leq \Aut(K/F)$.
    \end{enumerate}
\end{lemma}

\eg $\Q$ is an intermediate extensions of $\Q(\sqrt[3]{2})/\Q$. Then $\Aut(\Q/\Q)=\{\mathbb{1}\}$. We further observe that since automorphisms permute roots, $\Aut(\Q(\sqrt[3]{2})/\Q) = \{\mathbb{1}\}$. Hence $\Aut(\Q/\Q)=\Aut(\Q(\sqrt[3]{2})/\Q)$ and so the fixed field by $\Q(\sqrt[3]{2})$ is given by $\Q(\sqrt[3]{2})_{\Aut(\Q/\Q)}=\Q(\sqrt[3]{2})$. We note that $\Q(\sqrt[3]{2})$ is not Galois!
\sidenote{Roots of $x^3-2$ are $\sqrt[3]{2}, \zeta_3 \sqrt[3]{2}, \zeta_3^2 \sqrt[3]{2}$, so two roots are not in $\Q(\sqrt[3]{2})$, and so $\sqrt[3]{2}$ could be only mapped to itself.}


\begin{theorem}[The Fundamental Theorem of Galois Theory]
    \addlink{Fundamental Theorem of Galois Theory} If $K/F$ is a (finite) Galois extension, then the maps $H\mapsto K_H$ and $E\mapsto \Aut(K/E)$ gives an \textit{inclusion-reversing} \textbf{bijection} between the subgroups of $\Aut(K/F)$ and the intermediate extensions of $K/F$. \sidenote{We see if $K/F$ is Galois, then $K/E$ is also Galois as if $K$ is the splitting field of some poly in $F$, then it's certainly true for $E$.}
    
    Furthermore, $[K:E]=|\Aut(K/E)|$, and $$[E:F] = |\Aut(K/F):\Aut(K/E)| = |\Aut(K/F)|/|\Aut(K/E)|$$

    Moreover, $E/F$ is Galois \ifnif $\Aut(K/E)$ is a \textbf{normal subgroup} of $\Aut(K/F)$, in which case $$\Aut(E/F)=\Aut(K/F)/\Aut(K/E)$$

    In other words,
    \[\begin{tikzcd}[ampersand replacement=\&,cramped,column sep=large,row sep=2.25em]
        K \&\&\& 1 \\
        E \&\&\& {H=\Aut(K/E)} \\
        F \&\&\& {\mathrm{Aut}(K/F)}
        \arrow["{\{\alpha\in K\, \mid\, \sigma(\alpha)=\alpha\; \forall\; \sigma\in H\}}"', shift right=2, from=2-4, to=2-1]
        \arrow["{\mathrm{Aut}(K/E)}"', shift right=2, from=2-1, to=2-4]
        \arrow[no head, from=1-1, to=2-1]
        \arrow[no head, from=2-1, to=3-1]
        \arrow[no head, from=1-4, to=2-4]
        \arrow[no head, from=2-4, to=3-4]
    \end{tikzcd}\]
    where $H$ is a subgroup of $\Aut(K/F)$ that fixes the field $E$, an extension of $F$ contained in $K$. If $H\nsub \Aut(K/F)$, then $E/F$ is normal.
\end{theorem}

\eg \addlink{Example of Galois correspondence}Consider $\Q(\zeta_3, \sqrt[3]{2})/\Q$, the splitting field extension of $x^3-2$.\footnote{Suppose $\zeta$ is a primitive $n$th root of unity; then so is $\zeta^k$ \ifnif the gcd $(k,n)=1$, i.e. $k,n$ are relatively prime.} \sidenote{$|\zeta_3|=3$, a primitive 3rd root of unity.}
\[\begin{tikzcd}[ampersand replacement=\&,cramped]
	{\mathbb{Q}(\zeta_3,\sqrt[3]2)} \\
	{\mathbb{Q}(\sqrt[3]2)} \\
	\& {\mathbb Q(\zeta_3)} \\
	{\mathbb Q}
	\arrow["2", no head, from=1-1, to=2-1]
	\arrow["3", no head, from=2-1, to=4-1]
	\arrow["3", no head, from=1-1, to=3-2]
	\arrow["2"', no head, from=4-1, to=3-2]
	\arrow["{(x^3-1)}"', draw=none, from=2-1, to=4-1]
	\arrow["{(x^2+x+1)}"', draw=none, from=1-1, to=2-1]
\end{tikzcd}\]
Let $\sigma,\tau \in \Aut(\mathbb{Q}(\zeta_3,\sqrt[3]2)/\Q)$ be given by:
\begin{center}
    \sidenote{Isomorphisms preserve order, so they \textbf{must} take an $n$th root of unity to another $n$th root of unity!}
    \begin{tabular}[t]{ll}
        $\sigma: \left\{ \begin{array}{cl}
            \sqrt[3]2 \mapsto \zeta_3 \sqrt[3]2\\
                    \zeta_3 \mapsto \zeta_3
            \end{array} \right.$ 
            & $\tau: \left\{ \begin{array}{cl}
                \sqrt[3]2 \mapsto \sqrt[3]2\\
                        \zeta_3 \mapsto \zeta_3^2
                \end{array} \right.$
    \end{tabular}
\end{center}
So $\genset{\sigma,\tau\mid \sigma^3=\tau^2=id, \sigma\tau=\tau\sigma^2}\cong S_3$. 

Since this is a subgroup of $\Gal(\mathbb{Q}(\zeta_3,\sqrt[3]2)/\Q)$ that has the same finite order of 6, this must just be $\Gal(\mathbb{Q}(\zeta_3,\sqrt[3]2)/\Q)$ itself; hence, $\Gal(\mathbb{Q}(\zeta_3,\sqrt[3]2)/\Q)\cong S_3$.

Now we look at the subgroups of $S_3$ (in reverse):
\[\begin{tikzcd}[ampersand replacement=\&,cramped,sep=tiny]
	\& 1 \\
	\&\& {\langle \tau\rangle} \& {\langle \sigma\tau\rangle} \& {\langle \sigma^2\tau\rangle} \\
	\langle\sigma\rangle \\
	\& {\mathrm{Gal}(\mathbb Q(\zeta_3,\sqrt[3]2)/\mathbb Q)}
	\arrow["3", no head, from=1-2, to=3-1]
	\arrow[no head, from=1-2, to=2-3]
	\arrow[no head, from=1-2, to=2-4]
	\arrow["2", no head, from=1-2, to=2-5]
	\arrow["2", no head, from=3-1, to=4-2]
    \arrow["{\nsub}"', no head, from=3-1, to=4-2]
	\arrow[no head, from=2-3, to=4-2]
	\arrow[no head, from=2-4, to=4-2]
	\arrow["3", no head, from=2-5, to=4-2]
\end{tikzcd}\]
And then think about the \textbf{fixed field} of each subgroup correspondingly:\sidenote{Remember normal extn means `is splitting field', i.e. `one root -> all roots'}
\[\begin{tikzcd}[ampersand replacement=\&,cramped,sep=scriptsize]
	\& {\mathbb Q(\zeta_3,\sqrt[3]2)} \\
	\&\& {\mathbb Q(\sqrt[3]2)} \& {\mathbb Q(\zeta_3^2\sqrt[3]2)} \& {\mathbb Q(\zeta_3\sqrt[3]2)} \\
	{\mathbb Q(\zeta_3)} \\
	\& {\mathbb Q}
	\arrow[no head, from=1-2, to=3-1]
	\arrow[no head, from=1-2, to=2-3]
	\arrow[no head, from=1-2, to=2-4]
	\arrow[no head, from=1-2, to=2-5]
	\arrow["{\deg 2}", no head, from=3-1, to=4-2]
    \arrow["{\text{normal extn}}"', no head, from=3-1, to=4-2]
	\arrow["{\deg 3}"', no head, from=2-3, to=4-2]
	\arrow[no head, from=2-4, to=4-2]
	\arrow[no head, from=2-5, to=4-2]
\end{tikzcd}\]
Note that the normal extension corresponds to the normal subgroup!\sidenote{$\Q(\sqrt[3]{2})$ is not Galois!}

\spl

\begin{proposition}[Primitive Element Theorem]\label[prop]{prop:finite-galois-simple}\addlink{Finite Galois extensions are simple}
    If $K/F$ is a \textbf{finite Galois} extension, then $K=F(\alpha)$ for some $\alpha\in K$.
\end{proposition}

\defn $K=F(\alpha)$ is a \textbf{simple} extension of $F$ and $\alpha$ is a primitive element.

\begin{proof}
    We first assume $|F|=\infty$.

    Recall that $K/F$ is finite \ifnif $K=F(\alpha_1,\alpha_2,\dots,\alpha_n)$ where $\alpha_i$ is algebraic over $F$. We will proceed by induction on $n$, whose base case $n=1$ gives a simple extension $F(\alpha_1)/F$.

    Recursive case: assume that for some $k\geq 1$ we have $F(\alpha_1,\dots,\alpha_k)$ being a simple extension $F(\alpha)$. Let $K/F$ be Galois and $K=F(\alpha_1,\dots, \alpha_{k-1}, \alpha,\beta)$. 
    
    Let $E=F(\alpha_1,\dots, \alpha_{k-1})$ \sidenote{such that $K=E(\alpha,\beta)$}and consider the intermediate family of extensions $\{E(\alpha+t\beta)\mid t\in F\}$. Since $|\Gal(K/F)|<\infty$ as we are talking about finite Galois extensions, there are \dashuline{finitely many distinct such extensions}, so $E(\alpha+t_1\beta)=E(\alpha+t_2\beta)$ for some $t_1\neq t_2$.

    Now we see that $\alpha+t_1\beta$ and $\alpha+t_2\beta$ must be in the same field $E(\alpha+t_1\beta)$. Hence, $(\alpha+t_1\beta)-(\alpha+t_2\beta)$ are in the field, so $(t_1-t_2)^{-1}\left((\alpha+t_1\beta)-(\alpha+t_2\beta)\right)=\beta$ is also in the field. Similarly, $\alpha\in E(\alpha+t_1\beta)$. Therefore, $K=E(\alpha,\beta)=E(\alpha+t_1\beta)=F(\alpha_1,\dots,\alpha_{k-1}, \alpha+t_1\beta)$, which has $k$ elements adjoined and is therefore simple.
\end{proof}

\rmk Above is true for char 0 fields even without the `Galois' hypothesis.
\addlink{Finite extensions for char 0 fields are simple}
\begin{proof}[Proof outline]
    Since $K$ is a finite extension of $F$ with $[K:F]<\infty$, then there must be $K=F(\alpha_1,\dots,\alpha_n)$ for some algebraic $\alpha_1,\dots,\alpha_n\in K$.\sidenote{Idea: if anything is not Galois, we add enough things to it to make it Galois!}

    Let $E$ be the splitting field of $\prod_{i=1}^{n}m_{\alpha_i,F}(x)$. We call $E$ the \textbf{Galois closure} of $K$ over $F$. Now, we have $E\supset K \supset F$ and $E/F$ is Galois. Thus, there are finitely many intermediate fields between $K$ and $F$. We can then use a similar proof for \cref{prop:finite-galois-simple}.
\end{proof}

\subsection{Cyclotomic fields}
\defn \addlink{Primitive roots of unity}Suppose $K$ is a field. An element $z\in K$ is called an $n$-th root of unity if $z^n=1$; and $z$ is a \textbf{primitive} $n$-th root of unity if $z^k\neq 1$ for any $1\leq k\leq n-1$.

\rmk $z$ is a \textbf{primitive} $n$-th root of unity \ifnif $|z|=n$ in $K^{\times}=K\backslash \{0\}$.
\rmk $z$ is an $n$-th root of unity \ifnif $z$ is a root of $x^n-1$.

\begin{lemma}
    If $K$ is a field containing one primitive $n$-th root of unity, then $K$ contains exactly $n$ roots of unity, exactly $\homo{n}$ of which are primitive.\sidenote{$\homo{n}$ is Euler's totient function, the count of integers $< n$ that are relatively prime to $n$.}
\end{lemma}

\rmk Recall:
\begin{itemize}
    \item $z^n=1$ \ifnif $|z|$ divides $n$.
    \item $|z^m|=\frac{|z|}{\gcd(m,|z|)}$.
    \item $|z^m|=|z|$ \ifnif $(m,|z|)=1$.
\end{itemize}
\eg If $|z|=10$, then $|z^6|=5=\frac{10}{(6,10)}$
\begin{proof}
    If $z\in K$ is a primitive $n$-th root of unity, then every element of $\genset{z}$ is an $n$-th root of unity, and so a root of $x^n-1$; but $|\genset{z}|=|z|=n$, so the subgroup $\genset{z}$ generated by $z$  must consist of all of the $n$ roots of $x^n-1$. Furthermore, $z^m$ is also a primitive $n$-th root of unity \ifnif $(m,n)=1$; thus, there are exactly $\homo{n}$ such elements in $\genset{z}$.
\end{proof}

In $\C$, we have $e^{i\frac{2\pi}{n}}$ is a primitive $n$-th root of unity. Suppose $\zeta_n\in \C$ is a primitive $n$-th root of unity.

\addlink{Cyclotomic polynomial}
\defn The \textbf{cyclotomic polynomial} is given by $$\Phi_n(x)=\underset{(n,k)=1}{\prod_{{0\leq k <n}}}(x-\zeta_n^k)$$
Properties: \begin{itemize}
    \item $x^n-1=\prod_{0\leq k <n}(x-\zeta_n^k)=\prod_{d|n}\prod_{\underset{(n,k)=d}{0\leq k<n}}(x-\zeta_n^k)=\prod_{d|n}\Phi_d(x)$
    \item $\deg(\Phi_n(x))=\homo{n}$
    \item $n=\sum_{d|n}\homo{d}$
    \eg $6=\homo{1}+\homo{2}+\homo3+\homo6$
\end{itemize}

\ontangent{
    \rmk If $K$ is a finite field, then $K^{\times}=\genset{z}$.
    \begin{proof}
        Since $K$ is finite, it must be an extension of $\F_p=\znz{\textit{p}}$ for some prime $p$; so $|K^{\times}| = p^n-1$ since $|K|=p^n$. For $d|p^n-1$, let $\mathcal{O}_d=\{z\in K^{\times}\mid |z|=d\}$. Observe that $K^{\times}=\bigcup_{d|p^n-1}\mathcal{O}_d$ is a disjoint union, and so $$|K^{\times}|=\sum_{d|p^n-1}|\mathcal{O}_d|=p^n-1=\sum_{d|p^n-1}\homo{d}$$
        which implies that $|\mathcal{O}_d|=\homo{d}$ for all $d|p^n-1$. Hence, in particular, $\mathcal{O}_{p^n-1}$ is nonempty, so any $z\in \mathcal{O}_{p^n-1}$ generates $K^{\times}$.
    \end{proof}

    \rmk This implies that the Primitive Element Theorem (see \cref{prop:finite-galois-simple}) is also true for finite fields.

}

\rmk $\Phi _n(x)$ is irreducible over $\Q$, and so it is the minimal polynomial $m_{\zeta_n, \Q}(x)$.

\lemma \label[lemma]{lemma:between-every-two-roots-exists-a-Galois-isomorphism}Suppose\sidenote{\begin{tikzcd}[ampersand replacement=\&,cramped,sep=tiny]
	{\sigma: K} \& K \\
	{F(\alpha)} \& {F(\beta)} \\
	{\text{id}: F} \& F
	\arrow[from=3-1, to=3-2]
	\arrow[from=2-1, to=2-2]
	\arrow[from=1-1, to=1-2]
	\arrow[no head, from=1-1, to=2-1]
	\arrow[no head, from=2-1, to=3-1]
	\arrow[no head, from=1-2, to=2-2]
	\arrow[no head, from=2-2, to=3-2]
\end{tikzcd}} $F$ is a field where all irreducible polynomials are separable. \addlink{Between every two roots exists a Galois isomorphism}
Let $p(x)\in F[x]$ irreducible and split completely in a Galois extension $K/F$, and let $\alpha,\beta$ be two roots of $p(x)$. Then there exists $\sigma\in \Gal(K/F)$ such that $\sigma(\alpha)=\beta$.

\addlink{Cyclotomic extension}
\defn The extension $\Q(\zeta_n)/\Q$ is called the $n$-th \textbf{cyclotomic extension} of $\Q$.
\rmk{All other primitive $n$-th roots of unity are of the form $\zeta_n^a$ w/ $(a,n)=1$, so $\Q(\zeta_n)$ is the splitting field of $\Phi_n(x)$ over $\Q$, so the extension is Galois.}

If $\sigma\in \Gal(\Q(\zeta_n)/\Q)$, then $\sigma$ is completely determined by $\sigma(\zeta_n)$. But $\sigma(\zeta_n)$ must be another primitive $n$-th root of unity, so $\sigma(\zeta_n)$ must be $\zeta_n^a$ for some $0<a<n$ with $(a,n)=1$. 
\addlink{Galois groups of cyclotomic fields are $\Z/n\Z$}

Moreover, by \cref{lemma:between-every-two-roots-exists-a-Galois-isomorphism}, each $a$ corresponds to a Galois automorphism $\sigma$; in fact, the map $\Gal(\Q(\zeta_n)/\Q)\xrightarrow{\sim} (\Z/n\Z)^\times$ where $\sigma\mapsto \bar a$ is a \textbf{group isomorphism}.

\addlink{Abelian Galois extension}
\defn A Galois extension $K/F$ is abelian if $\Gal(K/F)$ is abelian.\sidenote{All finite field extensions are abelian}

\begin{corollary}
    $\Q(\zeta_n)/\Q$ is abelian.
\end{corollary}

\begin{theorem}[Kronecker-Wober]\addlink{All abelian extensions are conatined in cyclotomic extensions}
    If $K/\Q$ is abelian, then $K\subseteq \Q(\zeta_n)$ for some $n$.
\end{theorem}

\subsection{Radical extensions and soluble groups}

\eg Suppose $K$ is the splitting field of $x^4-2$ over $\Q$. Then $K=\Q(\sqrt[4]{2},i)$\sidenote{since $\zeta_4=i$}.

Let\sidenote{{(not showing all intermed extns)}

\begin{tikzcd}[ampersand replacement=\&,cramped, sep=tiny]
	\& K \\
	\&\& {\mathbb Q(\sqrt[4]2)} \\
	{\mathbb Q(i)} \\
	\& {\mathbb Q} \& {}
	\arrow[from=3-1, to=4-2]
	\arrow["\genset{\sigma}", from=1-2, to=3-1]
	\arrow[from=1-2, to=2-3]
	\arrow[from=2-3, to=4-2]
\end{tikzcd}} 
$\sigma:\begin{cases}
    \sqrt[4]2\mapsto i\sqrt[4]2\\
    i\mapsto i
\end{cases}$ and $\sigma:\begin{cases}
    \sqrt[4]2\mapsto \sqrt[4]2\\
    i\mapsto -i
\end{cases}$. \\We look at the subgroup generated by $\sigma,\tau$: $$\genset{\sigma,\tau\mid \sigma^4=\tau^2=1, \sigma\tau=\tau\sigma^3}\cong D_8$$
Since $[K/\Q]=8$, we have found the Galois group of $K/\Q$.
 %%%%%%%%%%%%%%%%%%%%%%%%%    
\end{document} 